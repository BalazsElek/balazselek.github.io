\documentclass[12pt]{article}
\usepackage{amsmath,amsthm,amssymb,color,enumerate}
\usepackage{graphicx}
\usepackage[backref=page]{hyperref}
\usepackage{cleveref}
\usepackage[all]{xy}
%\usepackage{ytableau}
%\usepackage{palatino}
\usepackage{mathpazo}
\usepackage{euler}
\usepackage{subcaption}

\hypersetup{colorlinks=true,linkcolor=blue,citecolor=red}
\voffset=-10mm %to print: dvips *.dvi; ps2pdf *.ps
\oddsidemargin=-20pt
\headheight=-20pt     \topmargin=0pt
%\textheight=658pt   \textwidth=495pt %return to this if possible
%\textheight=658pt   \textwidth=499.5pt
%textheight=659pt   \textwidth=499.5pt
\textheight=700pt   \textwidth=499.5pt
%parskip=.3pt plus 1pt
\parskip=.2pt plus .5pt
%%%%%%%%%%%%%%%%%%%%%%%%%%%%%%%%%%%%%%%%%%%%%%%%%%%%%%%%%%%%%%%%%%%%
\newcommand{\fg}{\mathfrak{g}}
\newcommand{\fh}{\mathfrak{h}}
\newcommand{\fb}{\mathfrak{b}}
\newcommand{\fn}{\mathfrak{n}}
\newcommand{\fm}{\mathfrak{m}}
\newcommand{\bx}{\mathbf{x}}
\newcommand{\bu}{\mathbf{u}}
\newcommand{\by}{\mathbf{y}}
\newcommand{\bz}{\mathbf{z}}
\newcommand{\bn}{\mathbf{n}}
\newcommand{\br}{\mathbf{r}}
\newcommand{\ba}{\mathbf{a}}
\newcommand{\bv}{\mathbf{v}}
\newcommand{\cA}{\mathcal{A}}
\newcommand{\cB}{\mathcal{B}}
\newcommand{\cC}{\mathcal{C}}
\newcommand{\cE}{\mathcal{E}}
\newcommand{\cF}{\mathcal{F}}
\newcommand{\cG}{\mathcal{G}}
\newcommand{\cH}{\mathcal{H}}
\newcommand{\cI}{\mathcal{I}}
\newcommand{\cK}{\mathcal{K}}
\newcommand{\cL}{\mathcal{L}}
\newcommand{\cN}{\mathcal{N}}
\newcommand{\cS}{\mathcal{S}}
\newcommand{\cT}{\mathcal{T}}
\newcommand{\bA}{\mathbb{A}}
\newcommand{\bC}{\mathbb{C}}
\newcommand{\bD}{\mathbb{D}}
\newcommand{\bR}{\mathbb{R}}
\newcommand{\bF}{\mathbb{F}}
\newcommand{\bN}{\mathbb{N}}
\newcommand{\bS}{\mathbb{S}}
\newcommand{\cO}{\mathcal{O}}
\newcommand{\fF}{\mathfrak{F}}
\newcommand{\bP}{\mathbb{P}}
\newcommand{\cY}{\mathcal{Y}}
\newcommand{\cZ}{\mathcal{Z}}
\newcommand{\bZ}{\mathbb{Z}}
\newcommand{\fZ}{\mathfrak{Z}}
\renewcommand{\sl}{\mathfrak{sl}}
\newcommand{\gl}{\mathfrak{gl}}
\newcommand{\Mat}{\mathrm{Mat}}
\newcommand{\sslash}{\mathbin{/\mkern-6mu/}}
\DeclareMathOperator{\Hom}{Hom}
\DeclareMathOperator{\Mor}{Mor}
\DeclareMathOperator{\End}{End}
\DeclareMathOperator{\Tor}{Tor}
\DeclareMathOperator{\Ext}{Ext}
\DeclareMathOperator{\Spec}{Spec}
\DeclareMathOperator{\Proj}{Proj}
\DeclareMathOperator{\Sym}{Sym}
\DeclareMathOperator{\rank}{rank}
\DeclareMathOperator{\supp}{supp}
\DeclareMathOperator{\Loc}{Loc}
\DeclareMathOperator{\Gr}{Gr}
\DeclareMathOperator{\coker}{coker}
\DeclareMathOperator{\im}{im}
\DeclareMathOperator{\ch}{ch}
\DeclareMathOperator{\ad}{ad}
\DeclareMathOperator{\rk}{rk}
\DeclareMathOperator{\gr}{gr}
\DeclareMathOperator{\wt}{wt}
\DeclareMathOperator{\Res}{Res}
\DeclareMathOperator{\Rep}{Rep}
\DeclareMathOperator{\Ind}{Ind}
\DeclareMathOperator{\Tr}{Tr}
\DeclareMathOperator{\Lie}{Lie}
\DeclareMathOperator{\Vect}{Vect}
\DeclareMathOperator{\Span}{Span}
%%%%%%%%%%%%%%%%%%%%%%%%%%%%%%%%%%%%%%%%%%%%%%%%%%%%%%%%%%%%%%%%%%%%%
\newcommand\restr[2]{{% we make the whole thing an ordinary symbol
  \left.\kern-\nulldelimiterspace % automatically resize the bar with \right
  #1 % the function
  \vphantom{\big|} % pretend it's a little taller at normal size
  \right|_{#2} % this is the delimiter
  }}
%%%%%%%%%%%%%%%%%%%%%%%%%%%%%%%%%%%%%%%%%%%%%%%%%%%%%%%%%%%%%%%%%%%%%

\newtheorem{theorem}{Theorem}[section]
\newtheorem*{theorem*}{Theorem}
\newtheorem{proposition}[theorem]{Proposition}
\newtheorem*{proposition*}{Proposition}
\newtheorem{conjecture}[theorem]{Conjecture}
\newtheorem*{conjecture*}{Conjecture}
\newtheorem{lemma}[theorem]{Lemma}
\newtheorem*{lemma*}{Lemma}
\newtheorem{corollary}[theorem]{Corollary}
\newtheorem*{corollary*}{Corollary}
\newtheorem{definition}[theorem]{Definition}
\newtheorem*{definition*}{Definition}
\newtheorem{example}[theorem]{Example}
\newtheorem*{example*}{Example}
\newtheorem{exercise}[theorem]{Exercise}
\newtheorem*{exercise*}{Exercise}
\newtheorem{remark}[theorem]{Remark}
\newtheorem*{remark*}{Remark}
\newtheorem{question}[theorem]{Question}
\newtheorem*{question*}{Question}
\newtheorem{claim}[theorem]{Claim}
\newtheorem*{claim*}{Claim}

% \usepackage{amsmath,amsthm,amssymb,color,enumerate}
\usepackage{graphicx}
\usepackage[backref=page]{hyperref}
\usepackage{cleveref}
\usepackage[all]{xy}
%\usepackage{ytableau}
%\usepackage{palatino}
\usepackage{mathpazo}
\usepackage{euler}
\usepackage{subcaption}

\hypersetup{colorlinks=true,linkcolor=blue,citecolor=red}
\voffset=-10mm %to print: dvips *.dvi; ps2pdf *.ps
\oddsidemargin=-20pt
\headheight=-20pt     \topmargin=0pt
%\textheight=658pt   \textwidth=495pt %return to this if possible
%\textheight=658pt   \textwidth=499.5pt
%textheight=659pt   \textwidth=499.5pt
\textheight=700pt   \textwidth=499.5pt
%parskip=.3pt plus 1pt
\parskip=.2pt plus .5pt
%%%%%%%%%%%%%%%%%%%%%%%%%%%%%%%%%%%%%%%%%%%%%%%%%%%%%%%%%%%%%%%%%%%%
\newcommand{\fg}{\mathfrak{g}}
\newcommand{\fh}{\mathfrak{h}}
\newcommand{\fb}{\mathfrak{b}}
\newcommand{\fn}{\mathfrak{n}}
\newcommand{\fm}{\mathfrak{m}}
\newcommand{\bx}{\mathbf{x}}
\newcommand{\bu}{\mathbf{u}}
\newcommand{\by}{\mathbf{y}}
\newcommand{\bz}{\mathbf{z}}
\newcommand{\bn}{\mathbf{n}}
\newcommand{\br}{\mathbf{r}}
\newcommand{\ba}{\mathbf{a}}
\newcommand{\bv}{\mathbf{v}}
\newcommand{\cA}{\mathcal{A}}
\newcommand{\cB}{\mathcal{B}}
\newcommand{\cC}{\mathcal{C}}
\newcommand{\cE}{\mathcal{E}}
\newcommand{\cF}{\mathcal{F}}
\newcommand{\cG}{\mathcal{G}}
\newcommand{\cH}{\mathcal{H}}
\newcommand{\cI}{\mathcal{I}}
\newcommand{\cK}{\mathcal{K}}
\newcommand{\cL}{\mathcal{L}}
\newcommand{\cN}{\mathcal{N}}
\newcommand{\cS}{\mathcal{S}}
\newcommand{\cT}{\mathcal{T}}
\newcommand{\bA}{\mathbb{A}}
\newcommand{\bC}{\mathbb{C}}
\newcommand{\bD}{\mathbb{D}}
\newcommand{\bR}{\mathbb{R}}
\newcommand{\bF}{\mathbb{F}}
\newcommand{\bN}{\mathbb{N}}
\newcommand{\bS}{\mathbb{S}}
\newcommand{\cO}{\mathcal{O}}
\newcommand{\fF}{\mathfrak{F}}
\newcommand{\bP}{\mathbb{P}}
\newcommand{\cY}{\mathcal{Y}}
\newcommand{\cZ}{\mathcal{Z}}
\newcommand{\bZ}{\mathbb{Z}}
\newcommand{\fZ}{\mathfrak{Z}}
\renewcommand{\sl}{\mathfrak{sl}}
\newcommand{\gl}{\mathfrak{gl}}
\newcommand{\Mat}{\mathrm{Mat}}
\newcommand{\sslash}{\mathbin{/\mkern-6mu/}}
\DeclareMathOperator{\Hom}{Hom}
\DeclareMathOperator{\Mor}{Mor}
\DeclareMathOperator{\End}{End}
\DeclareMathOperator{\Tor}{Tor}
\DeclareMathOperator{\Ext}{Ext}
\DeclareMathOperator{\Spec}{Spec}
\DeclareMathOperator{\Proj}{Proj}
\DeclareMathOperator{\Sym}{Sym}
\DeclareMathOperator{\rank}{rank}
\DeclareMathOperator{\supp}{supp}
\DeclareMathOperator{\Loc}{Loc}
\DeclareMathOperator{\Gr}{Gr}
\DeclareMathOperator{\coker}{coker}
\DeclareMathOperator{\im}{im}
\DeclareMathOperator{\ch}{ch}
\DeclareMathOperator{\ad}{ad}
\DeclareMathOperator{\rk}{rk}
\DeclareMathOperator{\gr}{gr}
\DeclareMathOperator{\wt}{wt}
\DeclareMathOperator{\Res}{Res}
\DeclareMathOperator{\Rep}{Rep}
\DeclareMathOperator{\Ind}{Ind}
\DeclareMathOperator{\Tr}{Tr}
\DeclareMathOperator{\Lie}{Lie}
\DeclareMathOperator{\Vect}{Vect}
\DeclareMathOperator{\Span}{Span}
%%%%%%%%%%%%%%%%%%%%%%%%%%%%%%%%%%%%%%%%%%%%%%%%%%%%%%%%%%%%%%%%%%%%%
\newcommand\restr[2]{{% we make the whole thing an ordinary symbol
  \left.\kern-\nulldelimiterspace % automatically resize the bar with \right
  #1 % the function
  \vphantom{\big|} % pretend it's a little taller at normal size
  \right|_{#2} % this is the delimiter
  }}
%%%%%%%%%%%%%%%%%%%%%%%%%%%%%%%%%%%%%%%%%%%%%%%%%%%%%%%%%%%%%%%%%%%%%

\newtheorem{theorem}{Theorem}[section]
\newtheorem*{theorem*}{Theorem}
\newtheorem{proposition}[theorem]{Proposition}
\newtheorem*{proposition*}{Proposition}
\newtheorem{conjecture}[theorem]{Conjecture}
\newtheorem*{conjecture*}{Conjecture}
\newtheorem{lemma}[theorem]{Lemma}
\newtheorem*{lemma*}{Lemma}
\newtheorem{corollary}[theorem]{Corollary}
\newtheorem*{corollary*}{Corollary}
\newtheorem{definition}[theorem]{Definition}
\newtheorem*{definition*}{Definition}
\newtheorem{example}[theorem]{Example}
\newtheorem*{example*}{Example}
\newtheorem{exercise}[theorem]{Exercise}
\newtheorem*{exercise*}{Exercise}
\newtheorem{remark}[theorem]{Remark}
\newtheorem*{remark*}{Remark}
\newtheorem{question}[theorem]{Question}
\newtheorem*{question*}{Question}
\newtheorem{claim}[theorem]{Claim}
\newtheorem*{claim*}{Claim}

\usepackage{tikz}
\tikzset{>=latex}
\usetikzlibrary{calc}
\usetikzlibrary{backgrounds}
\usetikzlibrary{patterns,decorations.pathreplacing}
\usetikzlibrary{spy}
\usepackage{pgfplots}
\pgfplotsset{compat=1.12}
\usepgfplotslibrary{colormaps}
\usepgfplotslibrary{patchplots}
\usepgfplotslibrary{fillbetween}
\usepackage{array} % for \newcolumntype macro
\newcolumntype{C}{>{{}}c<{{}}} % for columns with binary operators
\newcommand\vv{\multicolumn{1}{c}{\vdots}}

\newcommand{\rubric}[1]{{\color{blue}{#1}}}

\begin{document}
\title{MATH223 Homework 6\\ (due Sunday, Nov/3, 11:59pm)}
\author{}
\date{}
\maketitle
\pagenumbering{gobble}

\begin{enumerate}
\item \label{qn:injective}(this is Exercise 14 in Axler 4) Suppose $p,q\in\mathcal{P}(\bC)$ are nonconstant polynomials with no zeros in common. Let $m=\deg p$ and $n=\deg q$. Use linear algebra as outlined below in (a)-(c) to prove that there exist $r\in \mathcal{P}_{n-1}(\bC)$ and $s\in\mathcal{P}_{m-1}(\bC)$ such that
  \[
    rp+sq=1.
  \]
  \begin{enumerate}[(a)]
  \item (2 marks) Define $T:\mathcal{P}_{n-1}(\bC)\times \mathcal{P}_{m-1}(\bC)\to \mathcal{P}_{m+n-1}(\bC)$ by
    \[
      T(r,s)=rp+sq.
    \]
    Show that the linear map $T$ is injective.
  \item (1 mark) Show that the linear map $T$ in (a) is surjective.
  \item (1 mark) Use (b) to conclude that there exist $r\in\mathcal{P}_{n-1}(\bC)$ and $s\in\mathcal{P}_{m-1}(\bC)$ such that \mbox{$rp+sq=1$}.
  \end{enumerate}
\item \label{qn:resultant} As in question \ref{qn:injective}, let $p(x)=a_0+a_1x+\ldots +a_mx^m$ and $q(x)=b_0+b_1x+\ldots +b_nx^n$ be nonconstant polynomials (with $a_m\neq 0$, $b_n\neq 0$), but in this question do not assume that they do not share a root. Let $T:\mathcal{P}_{n-1}(\bC)\times \mathcal{P}_{m-1}(\bC)\to \mathcal{P}_{m+n-1}(\bC)$ be the same linear map
  \[
    T(r,s)=rp+sq.    
  \]
  \begin{enumerate}[(a)]
  \item (1 mark) In question \ref{qn:injective} you showed that $T$ is an isomorphism if $p$ and $q$ do not share a root. Show the converse, i.e. prove that if $T$ is an isomorphism then $p$ and $q$ do not share a root.
  \item (2 marks) Recall that the list
    \[
      \left((1,0),(x,0),\ldots ,(x^{n-1},0), (0,1), (0,x), \ldots ,(0,x^{m-1})\right)
    \]
    is a basis of $\mathcal{P}_{n-1}(\bC)\times \mathcal{P}_{m-1}(\bC)$ and that
    \[
      (1,x,x^2,\ldots ,x^{m+n-1})
    \]
    is a basis of $\mathcal{P}_{m+n-1}(\bC)$. Compute the matrix of $T$ with respect to these bases.
  \item (2 marks) The expression $\det(\mathcal{M}(T))$ is called the \textbf{resultant} of the polynomials $p$ and $q$. If $\det(\mathcal{M}(T))=0$, then what can you say about the roots of $p$ and $q$? Justify your answer.
  \end{enumerate}
  \newpage
\item As in questions \ref{qn:injective} and \ref{qn:resultant}, let $p(x)=a_0+a_1x+\ldots +a_mx^m$ be a degree $m$ polynomial. Let $q(x)=\frac{d}{dx}(p(x))$ be the derivative of $p$. We say that \textbf{$p(x)$ has a double root at $\lambda$} if $p(x)=(x-\lambda)^2s(x)$ for some polynomial $s$ of degree $m-2$.
  \begin{enumerate}[(a)]
  \item (1 mark) Prove that $p(x)$ has a double root at $\lambda$ if and only if $p(\lambda)=q(\lambda)=0$ (\textbf{Hint:} This is a Calculus question).
  \item (3 marks) Let $m=2$ (i.e. $p$ is a quadratic polynomial). Use question \ref{qn:resultant} to find a condition on the coefficients of $p$ that is equivalent to $p$ having a double root (\textbf{Hint:} remember, $a_2\neq 0$). Do you recognize this expression? (\textbf{Hint:} It is generally best to leave the computation of determinants to a computer, especially if the matrix involved is large. There are numerous software that can compute determinants involving formal variables, \href{https://sagecell.sagemath.org/?z=eJyFVMFuozAQvSPxDyOhKlCcKFR7isQnVNr2irKRSUxiBY-JbRry9zs20LLqrjaHaGy9NzPvzZjkJBqJAtxFwEmcjRCW-SDtslKN0S0rkQG3IDHAbr2wTmqMI1W-xBH6vzhKpkQf3Ehet8JCo03AH7VoGnmUAp0FjqdwKfEknDBKInfCxlErrdMNt2VFCdIVPzzJFTzJLGSRvnS13WzUfj9D6xla_x2KHhoAw4qtHqssNKn4VQCHTrcP1EryFozEM9ylu_i27LL_NIAtPwvoqVtj3dg8d3AXcOfowGk6S7yCbhY5LdV6L39-nt-pRPr2xlSO-Q82K81nHXk1sMc-W5rIoenx6E0e6xntvE3BObLRSIqpJAefI46INUHSll2zXRwB_YxwvcG0rdbX3T5vq936OlWxnTjK5vF9OpTT3y2ksHHq0oIYuOpawQDiqAssW85aICFHuBHQCue8oV06ZCU_bHN-KJ6HfLPZUKSeh18qjm5_kGsie19vnlETo_5k1AckBoaWk7p38NA9HDkC9aVhIUL5QdTetpNsGmFIy0IDI5euIohgYUcmJTtIjVD6Y1z-ZE0ba7QKh1qcJaIXMjmCYqBp3zWlQj8Hmrt5kJnJ7ERVsBdW7CG5UPlRfpG_DPnwi15HMkuu1gXbfsGC5nUxgr6mP46l7RXSw6CAl1OVvNoulrygJV8Xe0rWSFrOiTJ3rDityRD4dXn7B1_9jz-u31Gj4xIt1Ae5tl8P6QvHoLfeLn8zbuLnBsfRa-kMR9tpK9IRnlbTunp17NvT9V3lC0j9DRIaz2gA9qLv6SsFHT1kl67oq0LHcsXGIPsN3bWYvg==&lang=sage&interacts=eJyLjgUAARUAuQ==}{sage} is one of them that is relatively easy to use.) 
  \item (2 marks) Let $m=3$ (i.e. $p$ is a cubic polynomial). Find a condition on the coefficients of $p$ that is equivalent to $p$ having a double root.
  \end{enumerate}
\item In this question we'll explore another application of resultants. We want to find solutions of the system
  \[
    \setlength{\arraycolsep}{0pt}
    \begin{array}{c C c C c C c C l}
      & & y^2& &  & - & (x^3+3x^2+2x+1) & = & 0 \\
      y^3 & + & y^2(3x+3) & + & y(3x^2+6x+2) & + & (x^3+3x^2+4x+2) & = & 0
    \end{array}
  \]
  of non-linear polynomial equations. That is, we want to find pairs $(x,y)$ of (real) numbers that satisfy both of the above equations.
  \begin{enumerate}[(a)]
  \item (3 marks) Define two polynomials
    \begin{align*}
      p(x,y)&=y^2-(x^3+3x^2+2x+1) \\
      q(x,y)&=y^3+y^2(3x+3)+y(3x^2+6x+2) +(x^3+3x^2+4x+2).
    \end{align*}
    Think of them as polynomials in the variable $y$ (so, we treat functions of $x$ as coefficients, notice that we already wrote them this way). Use question \ref{qn:resultant} to find a polynomial equation $s(x)$ in just the variable $x$ that is equivalent to $p(x,y)$ and $q(x,y)$ sharing a root.\footnote{This technique is one of the main tools of \href{https://en.wikipedia.org/wiki/Elimination_theory}{elimination theory}. Notice how much harder polynomial elimination is than Gaussian (linear) elimination.}
  \item (2 marks) Find a root of $s(x)$ and use this to find some solutions to the system\footnote{This polynomial will be high degree so we don't have a formula for its roots. Computers can also help with factorization sometimes, \href{https://sagecell.sagemath.org/?z=eJwrSyzSUK9Q1-TlSktMLskv0jDRqogzUdBVsADSxkDa0BjIMFLQVjDSqgCSxpoAWKMMaA==&lang=sage&interacts=eJyLjgUAARUAuQ==}{here is a sage example}.} (you should be able to find three of them).
  \end{enumerate}
\end{enumerate}


\bibliography{c:/Dropbox/TeX/mybibliography}{}
% \bibliography{/media/bazse/storage/Dropbox/TeX/mybibliography}{}
\bibliographystyle{alpha}

\end{document}
