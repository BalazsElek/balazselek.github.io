\documentclass[12pt]{article}
\usepackage{amsmath,amsthm,amssymb,color,enumerate}
\usepackage{graphicx}
\usepackage[backref=page]{hyperref}
\usepackage{cleveref}
\usepackage[all]{xy}
%\usepackage{ytableau}
%\usepackage{palatino}
\usepackage{mathpazo}
\usepackage{euler}
\usepackage{subcaption}

\hypersetup{colorlinks=true,linkcolor=blue,citecolor=red}
\voffset=-10mm %to print: dvips *.dvi; ps2pdf *.ps
\oddsidemargin=-20pt
\headheight=-20pt     \topmargin=0pt
%\textheight=658pt   \textwidth=495pt %return to this if possible
%\textheight=658pt   \textwidth=499.5pt
%textheight=659pt   \textwidth=499.5pt
\textheight=700pt   \textwidth=499.5pt
%parskip=.3pt plus 1pt
\parskip=.2pt plus .5pt
%%%%%%%%%%%%%%%%%%%%%%%%%%%%%%%%%%%%%%%%%%%%%%%%%%%%%%%%%%%%%%%%%%%%
\newcommand{\fg}{\mathfrak{g}}
\newcommand{\fh}{\mathfrak{h}}
\newcommand{\fb}{\mathfrak{b}}
\newcommand{\fn}{\mathfrak{n}}
\newcommand{\fm}{\mathfrak{m}}
\newcommand{\bx}{\mathbf{x}}
\newcommand{\bu}{\mathbf{u}}
\newcommand{\by}{\mathbf{y}}
\newcommand{\bz}{\mathbf{z}}
\newcommand{\bn}{\mathbf{n}}
\newcommand{\br}{\mathbf{r}}
\newcommand{\ba}{\mathbf{a}}
\newcommand{\bv}{\mathbf{v}}
\newcommand{\cA}{\mathcal{A}}
\newcommand{\cB}{\mathcal{B}}
\newcommand{\cC}{\mathcal{C}}
\newcommand{\cE}{\mathcal{E}}
\newcommand{\cF}{\mathcal{F}}
\newcommand{\cG}{\mathcal{G}}
\newcommand{\cH}{\mathcal{H}}
\newcommand{\cI}{\mathcal{I}}
\newcommand{\cK}{\mathcal{K}}
\newcommand{\cL}{\mathcal{L}}
\newcommand{\cN}{\mathcal{N}}
\newcommand{\cS}{\mathcal{S}}
\newcommand{\cT}{\mathcal{T}}
\newcommand{\bA}{\mathbb{A}}
\newcommand{\bC}{\mathbb{C}}
\newcommand{\bD}{\mathbb{D}}
\newcommand{\bR}{\mathbb{R}}
\newcommand{\bF}{\mathbb{F}}
\newcommand{\bN}{\mathbb{N}}
\newcommand{\bS}{\mathbb{S}}
\newcommand{\cO}{\mathcal{O}}
\newcommand{\fF}{\mathfrak{F}}
\newcommand{\bP}{\mathbb{P}}
\newcommand{\cY}{\mathcal{Y}}
\newcommand{\cZ}{\mathcal{Z}}
\newcommand{\bZ}{\mathbb{Z}}
\newcommand{\fZ}{\mathfrak{Z}}
\renewcommand{\sl}{\mathfrak{sl}}
\newcommand{\gl}{\mathfrak{gl}}
\newcommand{\Mat}{\mathrm{Mat}}
\newcommand{\sslash}{\mathbin{/\mkern-6mu/}}
\DeclareMathOperator{\Hom}{Hom}
\DeclareMathOperator{\Mor}{Mor}
\DeclareMathOperator{\End}{End}
\DeclareMathOperator{\Tor}{Tor}
\DeclareMathOperator{\Ext}{Ext}
\DeclareMathOperator{\Spec}{Spec}
\DeclareMathOperator{\Proj}{Proj}
\DeclareMathOperator{\Sym}{Sym}
\DeclareMathOperator{\rank}{rank}
\DeclareMathOperator{\supp}{supp}
\DeclareMathOperator{\Loc}{Loc}
\DeclareMathOperator{\Gr}{Gr}
\DeclareMathOperator{\coker}{coker}
\DeclareMathOperator{\im}{im}
\DeclareMathOperator{\ch}{ch}
\DeclareMathOperator{\ad}{ad}
\DeclareMathOperator{\rk}{rk}
\DeclareMathOperator{\gr}{gr}
\DeclareMathOperator{\wt}{wt}
\DeclareMathOperator{\Res}{Res}
\DeclareMathOperator{\Rep}{Rep}
\DeclareMathOperator{\Ind}{Ind}
\DeclareMathOperator{\Tr}{Tr}
\DeclareMathOperator{\Lie}{Lie}
\DeclareMathOperator{\Vect}{Vect}
\DeclareMathOperator{\Span}{Span}
%%%%%%%%%%%%%%%%%%%%%%%%%%%%%%%%%%%%%%%%%%%%%%%%%%%%%%%%%%%%%%%%%%%%%
\newcommand\restr[2]{{% we make the whole thing an ordinary symbol
  \left.\kern-\nulldelimiterspace % automatically resize the bar with \right
  #1 % the function
  \vphantom{\big|} % pretend it's a little taller at normal size
  \right|_{#2} % this is the delimiter
  }}
%%%%%%%%%%%%%%%%%%%%%%%%%%%%%%%%%%%%%%%%%%%%%%%%%%%%%%%%%%%%%%%%%%%%%

\newtheorem{theorem}{Theorem}[section]
\newtheorem*{theorem*}{Theorem}
\newtheorem{proposition}[theorem]{Proposition}
\newtheorem*{proposition*}{Proposition}
\newtheorem{conjecture}[theorem]{Conjecture}
\newtheorem*{conjecture*}{Conjecture}
\newtheorem{lemma}[theorem]{Lemma}
\newtheorem*{lemma*}{Lemma}
\newtheorem{corollary}[theorem]{Corollary}
\newtheorem*{corollary*}{Corollary}
\newtheorem{definition}[theorem]{Definition}
\newtheorem*{definition*}{Definition}
\newtheorem{example}[theorem]{Example}
\newtheorem*{example*}{Example}
\newtheorem{exercise}[theorem]{Exercise}
\newtheorem*{exercise*}{Exercise}
\newtheorem{remark}[theorem]{Remark}
\newtheorem*{remark*}{Remark}
\newtheorem{question}[theorem]{Question}
\newtheorem*{question*}{Question}
\newtheorem{claim}[theorem]{Claim}
\newtheorem*{claim*}{Claim}

%\usepackage{amsmath,amsthm,amssymb,color,enumerate}
\usepackage{graphicx}
\usepackage[backref=page]{hyperref}
\usepackage{cleveref}
\usepackage[all]{xy}
%\usepackage{ytableau}
%\usepackage{palatino}
\usepackage{mathpazo}
\usepackage{euler}
\usepackage{subcaption}

\hypersetup{colorlinks=true,linkcolor=blue,citecolor=red}
\voffset=-10mm %to print: dvips *.dvi; ps2pdf *.ps
\oddsidemargin=-20pt
\headheight=-20pt     \topmargin=0pt
%\textheight=658pt   \textwidth=495pt %return to this if possible
%\textheight=658pt   \textwidth=499.5pt
%textheight=659pt   \textwidth=499.5pt
\textheight=700pt   \textwidth=499.5pt
%parskip=.3pt plus 1pt
\parskip=.2pt plus .5pt
%%%%%%%%%%%%%%%%%%%%%%%%%%%%%%%%%%%%%%%%%%%%%%%%%%%%%%%%%%%%%%%%%%%%
\newcommand{\fg}{\mathfrak{g}}
\newcommand{\fh}{\mathfrak{h}}
\newcommand{\fb}{\mathfrak{b}}
\newcommand{\fn}{\mathfrak{n}}
\newcommand{\fm}{\mathfrak{m}}
\newcommand{\bx}{\mathbf{x}}
\newcommand{\bu}{\mathbf{u}}
\newcommand{\by}{\mathbf{y}}
\newcommand{\bz}{\mathbf{z}}
\newcommand{\bn}{\mathbf{n}}
\newcommand{\br}{\mathbf{r}}
\newcommand{\ba}{\mathbf{a}}
\newcommand{\bv}{\mathbf{v}}
\newcommand{\cA}{\mathcal{A}}
\newcommand{\cB}{\mathcal{B}}
\newcommand{\cC}{\mathcal{C}}
\newcommand{\cE}{\mathcal{E}}
\newcommand{\cF}{\mathcal{F}}
\newcommand{\cG}{\mathcal{G}}
\newcommand{\cH}{\mathcal{H}}
\newcommand{\cI}{\mathcal{I}}
\newcommand{\cK}{\mathcal{K}}
\newcommand{\cL}{\mathcal{L}}
\newcommand{\cN}{\mathcal{N}}
\newcommand{\cS}{\mathcal{S}}
\newcommand{\cT}{\mathcal{T}}
\newcommand{\bA}{\mathbb{A}}
\newcommand{\bC}{\mathbb{C}}
\newcommand{\bD}{\mathbb{D}}
\newcommand{\bR}{\mathbb{R}}
\newcommand{\bF}{\mathbb{F}}
\newcommand{\bN}{\mathbb{N}}
\newcommand{\bS}{\mathbb{S}}
\newcommand{\cO}{\mathcal{O}}
\newcommand{\fF}{\mathfrak{F}}
\newcommand{\bP}{\mathbb{P}}
\newcommand{\cY}{\mathcal{Y}}
\newcommand{\cZ}{\mathcal{Z}}
\newcommand{\bZ}{\mathbb{Z}}
\newcommand{\fZ}{\mathfrak{Z}}
\renewcommand{\sl}{\mathfrak{sl}}
\newcommand{\gl}{\mathfrak{gl}}
\newcommand{\Mat}{\mathrm{Mat}}
\newcommand{\sslash}{\mathbin{/\mkern-6mu/}}
\DeclareMathOperator{\Hom}{Hom}
\DeclareMathOperator{\Mor}{Mor}
\DeclareMathOperator{\End}{End}
\DeclareMathOperator{\Tor}{Tor}
\DeclareMathOperator{\Ext}{Ext}
\DeclareMathOperator{\Spec}{Spec}
\DeclareMathOperator{\Proj}{Proj}
\DeclareMathOperator{\Sym}{Sym}
\DeclareMathOperator{\rank}{rank}
\DeclareMathOperator{\supp}{supp}
\DeclareMathOperator{\Loc}{Loc}
\DeclareMathOperator{\Gr}{Gr}
\DeclareMathOperator{\coker}{coker}
\DeclareMathOperator{\im}{im}
\DeclareMathOperator{\ch}{ch}
\DeclareMathOperator{\ad}{ad}
\DeclareMathOperator{\rk}{rk}
\DeclareMathOperator{\gr}{gr}
\DeclareMathOperator{\wt}{wt}
\DeclareMathOperator{\Res}{Res}
\DeclareMathOperator{\Rep}{Rep}
\DeclareMathOperator{\Ind}{Ind}
\DeclareMathOperator{\Tr}{Tr}
\DeclareMathOperator{\Lie}{Lie}
\DeclareMathOperator{\Vect}{Vect}
\DeclareMathOperator{\Span}{Span}
%%%%%%%%%%%%%%%%%%%%%%%%%%%%%%%%%%%%%%%%%%%%%%%%%%%%%%%%%%%%%%%%%%%%%
\newcommand\restr[2]{{% we make the whole thing an ordinary symbol
  \left.\kern-\nulldelimiterspace % automatically resize the bar with \right
  #1 % the function
  \vphantom{\big|} % pretend it's a little taller at normal size
  \right|_{#2} % this is the delimiter
  }}
%%%%%%%%%%%%%%%%%%%%%%%%%%%%%%%%%%%%%%%%%%%%%%%%%%%%%%%%%%%%%%%%%%%%%

\newtheorem{theorem}{Theorem}[section]
\newtheorem*{theorem*}{Theorem}
\newtheorem{proposition}[theorem]{Proposition}
\newtheorem*{proposition*}{Proposition}
\newtheorem{conjecture}[theorem]{Conjecture}
\newtheorem*{conjecture*}{Conjecture}
\newtheorem{lemma}[theorem]{Lemma}
\newtheorem*{lemma*}{Lemma}
\newtheorem{corollary}[theorem]{Corollary}
\newtheorem*{corollary*}{Corollary}
\newtheorem{definition}[theorem]{Definition}
\newtheorem*{definition*}{Definition}
\newtheorem{example}[theorem]{Example}
\newtheorem*{example*}{Example}
\newtheorem{exercise}[theorem]{Exercise}
\newtheorem*{exercise*}{Exercise}
\newtheorem{remark}[theorem]{Remark}
\newtheorem*{remark*}{Remark}
\newtheorem{question}[theorem]{Question}
\newtheorem*{question*}{Question}
\newtheorem{claim}[theorem]{Claim}
\newtheorem*{claim*}{Claim}


\usepackage{array} % for \newcolumntype macro
\newcolumntype{C}{>{{}}c<{{}}} % for columns with binary operators
\newcommand\vv{\multicolumn{1}{c}{\vdots}}

\begin{document}
\title{MATH223 Homework 3\\ (due Monday, September 11, 30:59pm)}
\author{}
\date{}
\maketitle
\pagenumbering{gobble}

\begin{enumerate}
\item In this question we will try to understand this week's new definitions by putting them into a more concrete context of matrices.
  \begin{enumerate}
  \item (2 marks) Show that $\vec v_1, \vec v_2, \ldots , \vec v_m$, with $\vec v_i\in \bF^n$ is linearly independent if and only if the RREF of the matrix whose columns are the vectors
    \[
      \left( \vec v_1 , \vec v_2, \ldots , \vec v_m \right)
    \]
    has $m$ pivots.
  \item (2 marks) Show that $\vec v_1, \vec v_2, \ldots , \vec v_m$, with $\vec v_i\in \bF^n$ spans $\bF^n$ if and only if the RREF of the matrix
    \[
      \left( \vec v_1 , \vec v_2, \ldots , \vec v_m \right)
    \]
    has $n$ pivots.
  \item (1 mark) Based on the previous parts, when does a list $\vec v_1, \vec v_2, \ldots , \vec v_m$, with $\vec v_i\in \bF^n$ form a basis for $\bF^n$? Give your answer in terms of the RREF of the matrix
    \[
      \left( \vec v_1 , \vec v_2, \ldots , \vec v_m \right)
    \]
  \end{enumerate}
\item (2 marks, this is Exercise 13 in Axler 2A) Suppose $\vec v_1,\ldots \vec v_m$ is linearly independent in $V$ and $w\in V$. Show that
  \[
    v_1,\ldots ,v_m,w\text{ is linearly independent} \Leftrightarrow w\not\in\Span(v_1,\ldots ,v_m).
  \]
\item Consider $\bR$ as a vector space over $\mathbb{Q}$ (the rational numbers). You do not need to verify that it is a vector space.
  \begin{enumerate}
  \item (1 mark) Prove that the list $(1,\sqrt{2})$ is linearly independent.
  \item (1 mark) Prove that the list $(1,\sqrt{2},\sqrt[3]{2})$ is linearly independent.
  \item (1 mark) Assuming that the list $(1,\sqrt{2},\sqrt[3]{2},\ldots ,\sqrt[n]{2})$ is linearly independent \mbox{(for any $n$)}, prove that $\bR$ is an infinite-dimensional vector space over $\mathbb{Q}$.
  \end{enumerate}
\item Recall that a binomial coefficient $\binom{n}{k}$ is defined as
  \[
    \binom{n}{k}=\frac{n!}{k!(n-k)!}=\frac{n(n-1)(n-2)\cdots (n-k+1)}{k(k-1)\cdots (2)(1)}.
  \]
  Similarly, we define, for a variable $x$ the following \emph{polynomial}:
  \[
    \binom{x}{k}=\frac{x(x-1)(x-2)\cdots (x-k+1)}{k!}.
  \]
  \begin{enumerate}
  \item (1 mark) Could we have instead defined this polynomial as
    \[
      \binom{x}{k}=\frac{x!}{k!(x-k)!}?
    \]
    Explain your answer carefully.
  \item (2 marks) Prove that the list
    \[
      \binom{x}{0}, \binom{x}{1}, \ldots , \binom{x}{m}
    \]
    is a basis for $\mathcal{P}_m(\bF)$.\footnote{We will see later that this basis is better for some purposes than $1,x,x^2,\ldots , x^m$. Here is something that you might find nice: what is $\frac{d}{dx}\binom{x}{k}$?}
  \end{enumerate}
\item (3 marks this is Exercise 15 in Axler 2C) Suppose $V$ is finite-dimensional and $V_1,V_2,V_3$ are subspaces of $V$ with $\dim V_1+\dim V_2+\dim V_3>2\dim V$. Prove that $V_1\cap V_2 \cap V_3\neq \{0\}$.
\item In this question we address a very commonly held false belief about dimensions.
  \begin{enumerate}
  \item \label{thm:inclexcl} (2 marks) Let $S_1, S_2, S_3$ be finite sets. Prove that
  \[
    |S_1\cup S_2\cup S_3|=|S_1|+|S_2|+|S_3|-|S_1\cap S_2|-|S_1\cap S_3|-|S_2\cap S_3|+|S_1\cap S_2\cap S_3|,
  \]
  where $|S|$ denotes the cardinality (number of elements) of a set. (\textbf{Hint:} this part has nothing to do with linear algebra).
\item (2 marks) this is Exercise 19 in Axler 2C) Let $U_1, U_2, U_3$ be subspaces. We have seen (or will see) in class that
  \[
    \dim(U_1+U_2)=\dim(U_1)+\dim(U_2)-\dim(U_1\cap U_2).
  \]
  You might guess (following part \ref{thm:inclexcl}) that this generalizes to
  \begin{align*}
    \dim(U_1+U_2+U_3)&=\dim(U_1)+\dim(U_2)+\dim(U_3)-\dim(U_1\cap U_2)-\dim(U_1\cap U_3)\\ &-\dim(U_2\cap U_3)+\dim(U_1\cap U_2\cap U_3)
  \end{align*}
  but this would be wrong. Find an example of three subspaces $U_1, U_2, U_3$ that do not satisfy the above equality.
  \end{enumerate}
\end{enumerate}


\bibliography{c:/Dropbox/TeX/mybibliography}{}
% \bibliography{/media/bazse/storage/Dropbox/TeX/mybibliography}{}
\bibliographystyle{alpha}

\end{document}