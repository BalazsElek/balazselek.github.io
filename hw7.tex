\documentclass[12pt]{article}
\usepackage{amsmath,amsthm,amssymb,color,enumerate}
\usepackage{graphicx}
\usepackage[backref=page]{hyperref}
\usepackage{cleveref}
\usepackage[all]{xy}
%\usepackage{ytableau}
%\usepackage{palatino}
\usepackage{mathpazo}
\usepackage{euler}
\usepackage{subcaption}

\hypersetup{colorlinks=true,linkcolor=blue,citecolor=red}
\voffset=-10mm %to print: dvips *.dvi; ps2pdf *.ps
\oddsidemargin=-20pt
\headheight=-20pt     \topmargin=0pt
%\textheight=658pt   \textwidth=495pt %return to this if possible
%\textheight=658pt   \textwidth=499.5pt
%textheight=659pt   \textwidth=499.5pt
\textheight=700pt   \textwidth=499.5pt
%parskip=.3pt plus 1pt
\parskip=.2pt plus .5pt
%%%%%%%%%%%%%%%%%%%%%%%%%%%%%%%%%%%%%%%%%%%%%%%%%%%%%%%%%%%%%%%%%%%%
\newcommand{\fg}{\mathfrak{g}}
\newcommand{\fh}{\mathfrak{h}}
\newcommand{\fb}{\mathfrak{b}}
\newcommand{\fn}{\mathfrak{n}}
\newcommand{\fm}{\mathfrak{m}}
\newcommand{\bx}{\mathbf{x}}
\newcommand{\bu}{\mathbf{u}}
\newcommand{\by}{\mathbf{y}}
\newcommand{\bz}{\mathbf{z}}
\newcommand{\bn}{\mathbf{n}}
\newcommand{\br}{\mathbf{r}}
\newcommand{\ba}{\mathbf{a}}
\newcommand{\bv}{\mathbf{v}}
\newcommand{\cA}{\mathcal{A}}
\newcommand{\cB}{\mathcal{B}}
\newcommand{\cC}{\mathcal{C}}
\newcommand{\cE}{\mathcal{E}}
\newcommand{\cF}{\mathcal{F}}
\newcommand{\cG}{\mathcal{G}}
\newcommand{\cH}{\mathcal{H}}
\newcommand{\cI}{\mathcal{I}}
\newcommand{\cK}{\mathcal{K}}
\newcommand{\cL}{\mathcal{L}}
\newcommand{\cN}{\mathcal{N}}
\newcommand{\cS}{\mathcal{S}}
\newcommand{\cT}{\mathcal{T}}
\newcommand{\bA}{\mathbb{A}}
\newcommand{\bC}{\mathbb{C}}
\newcommand{\bD}{\mathbb{D}}
\newcommand{\bR}{\mathbb{R}}
\newcommand{\bF}{\mathbb{F}}
\newcommand{\bN}{\mathbb{N}}
\newcommand{\bS}{\mathbb{S}}
\newcommand{\cO}{\mathcal{O}}
\newcommand{\fF}{\mathfrak{F}}
\newcommand{\bP}{\mathbb{P}}
\newcommand{\cY}{\mathcal{Y}}
\newcommand{\cZ}{\mathcal{Z}}
\newcommand{\bZ}{\mathbb{Z}}
\newcommand{\fZ}{\mathfrak{Z}}
\renewcommand{\sl}{\mathfrak{sl}}
\newcommand{\gl}{\mathfrak{gl}}
\newcommand{\Mat}{\mathrm{Mat}}
\newcommand{\sslash}{\mathbin{/\mkern-6mu/}}
\DeclareMathOperator{\Hom}{Hom}
\DeclareMathOperator{\Mor}{Mor}
\DeclareMathOperator{\End}{End}
\DeclareMathOperator{\Tor}{Tor}
\DeclareMathOperator{\Ext}{Ext}
\DeclareMathOperator{\Spec}{Spec}
\DeclareMathOperator{\Proj}{Proj}
\DeclareMathOperator{\Sym}{Sym}
\DeclareMathOperator{\rank}{rank}
\DeclareMathOperator{\supp}{supp}
\DeclareMathOperator{\Loc}{Loc}
\DeclareMathOperator{\Gr}{Gr}
\DeclareMathOperator{\coker}{coker}
\DeclareMathOperator{\im}{im}
\DeclareMathOperator{\ch}{ch}
\DeclareMathOperator{\ad}{ad}
\DeclareMathOperator{\rk}{rk}
\DeclareMathOperator{\gr}{gr}
\DeclareMathOperator{\wt}{wt}
\DeclareMathOperator{\Res}{Res}
\DeclareMathOperator{\Rep}{Rep}
\DeclareMathOperator{\Ind}{Ind}
\DeclareMathOperator{\Tr}{Tr}
\DeclareMathOperator{\Lie}{Lie}
\DeclareMathOperator{\Vect}{Vect}
\DeclareMathOperator{\Span}{Span}
%%%%%%%%%%%%%%%%%%%%%%%%%%%%%%%%%%%%%%%%%%%%%%%%%%%%%%%%%%%%%%%%%%%%%
\newcommand\restr[2]{{% we make the whole thing an ordinary symbol
  \left.\kern-\nulldelimiterspace % automatically resize the bar with \right
  #1 % the function
  \vphantom{\big|} % pretend it's a little taller at normal size
  \right|_{#2} % this is the delimiter
  }}
%%%%%%%%%%%%%%%%%%%%%%%%%%%%%%%%%%%%%%%%%%%%%%%%%%%%%%%%%%%%%%%%%%%%%

\newtheorem{theorem}{Theorem}[section]
\newtheorem*{theorem*}{Theorem}
\newtheorem{proposition}[theorem]{Proposition}
\newtheorem*{proposition*}{Proposition}
\newtheorem{conjecture}[theorem]{Conjecture}
\newtheorem*{conjecture*}{Conjecture}
\newtheorem{lemma}[theorem]{Lemma}
\newtheorem*{lemma*}{Lemma}
\newtheorem{corollary}[theorem]{Corollary}
\newtheorem*{corollary*}{Corollary}
\newtheorem{definition}[theorem]{Definition}
\newtheorem*{definition*}{Definition}
\newtheorem{example}[theorem]{Example}
\newtheorem*{example*}{Example}
\newtheorem{exercise}[theorem]{Exercise}
\newtheorem*{exercise*}{Exercise}
\newtheorem{remark}[theorem]{Remark}
\newtheorem*{remark*}{Remark}
\newtheorem{question}[theorem]{Question}
\newtheorem*{question*}{Question}
\newtheorem{claim}[theorem]{Claim}
\newtheorem*{claim*}{Claim}

% \usepackage{amsmath,amsthm,amssymb,color,enumerate}
\usepackage{graphicx}
\usepackage[backref=page]{hyperref}
\usepackage{cleveref}
\usepackage[all]{xy}
%\usepackage{ytableau}
%\usepackage{palatino}
\usepackage{mathpazo}
\usepackage{euler}
\usepackage{subcaption}

\hypersetup{colorlinks=true,linkcolor=blue,citecolor=red}
\voffset=-10mm %to print: dvips *.dvi; ps2pdf *.ps
\oddsidemargin=-20pt
\headheight=-20pt     \topmargin=0pt
%\textheight=658pt   \textwidth=495pt %return to this if possible
%\textheight=658pt   \textwidth=499.5pt
%textheight=659pt   \textwidth=499.5pt
\textheight=700pt   \textwidth=499.5pt
%parskip=.3pt plus 1pt
\parskip=.2pt plus .5pt
%%%%%%%%%%%%%%%%%%%%%%%%%%%%%%%%%%%%%%%%%%%%%%%%%%%%%%%%%%%%%%%%%%%%
\newcommand{\fg}{\mathfrak{g}}
\newcommand{\fh}{\mathfrak{h}}
\newcommand{\fb}{\mathfrak{b}}
\newcommand{\fn}{\mathfrak{n}}
\newcommand{\fm}{\mathfrak{m}}
\newcommand{\bx}{\mathbf{x}}
\newcommand{\bu}{\mathbf{u}}
\newcommand{\by}{\mathbf{y}}
\newcommand{\bz}{\mathbf{z}}
\newcommand{\bn}{\mathbf{n}}
\newcommand{\br}{\mathbf{r}}
\newcommand{\ba}{\mathbf{a}}
\newcommand{\bv}{\mathbf{v}}
\newcommand{\cA}{\mathcal{A}}
\newcommand{\cB}{\mathcal{B}}
\newcommand{\cC}{\mathcal{C}}
\newcommand{\cE}{\mathcal{E}}
\newcommand{\cF}{\mathcal{F}}
\newcommand{\cG}{\mathcal{G}}
\newcommand{\cH}{\mathcal{H}}
\newcommand{\cI}{\mathcal{I}}
\newcommand{\cK}{\mathcal{K}}
\newcommand{\cL}{\mathcal{L}}
\newcommand{\cN}{\mathcal{N}}
\newcommand{\cS}{\mathcal{S}}
\newcommand{\cT}{\mathcal{T}}
\newcommand{\bA}{\mathbb{A}}
\newcommand{\bC}{\mathbb{C}}
\newcommand{\bD}{\mathbb{D}}
\newcommand{\bR}{\mathbb{R}}
\newcommand{\bF}{\mathbb{F}}
\newcommand{\bN}{\mathbb{N}}
\newcommand{\bS}{\mathbb{S}}
\newcommand{\cO}{\mathcal{O}}
\newcommand{\fF}{\mathfrak{F}}
\newcommand{\bP}{\mathbb{P}}
\newcommand{\cY}{\mathcal{Y}}
\newcommand{\cZ}{\mathcal{Z}}
\newcommand{\bZ}{\mathbb{Z}}
\newcommand{\fZ}{\mathfrak{Z}}
\renewcommand{\sl}{\mathfrak{sl}}
\newcommand{\gl}{\mathfrak{gl}}
\newcommand{\Mat}{\mathrm{Mat}}
\newcommand{\sslash}{\mathbin{/\mkern-6mu/}}
\DeclareMathOperator{\Hom}{Hom}
\DeclareMathOperator{\Mor}{Mor}
\DeclareMathOperator{\End}{End}
\DeclareMathOperator{\Tor}{Tor}
\DeclareMathOperator{\Ext}{Ext}
\DeclareMathOperator{\Spec}{Spec}
\DeclareMathOperator{\Proj}{Proj}
\DeclareMathOperator{\Sym}{Sym}
\DeclareMathOperator{\rank}{rank}
\DeclareMathOperator{\supp}{supp}
\DeclareMathOperator{\Loc}{Loc}
\DeclareMathOperator{\Gr}{Gr}
\DeclareMathOperator{\coker}{coker}
\DeclareMathOperator{\im}{im}
\DeclareMathOperator{\ch}{ch}
\DeclareMathOperator{\ad}{ad}
\DeclareMathOperator{\rk}{rk}
\DeclareMathOperator{\gr}{gr}
\DeclareMathOperator{\wt}{wt}
\DeclareMathOperator{\Res}{Res}
\DeclareMathOperator{\Rep}{Rep}
\DeclareMathOperator{\Ind}{Ind}
\DeclareMathOperator{\Tr}{Tr}
\DeclareMathOperator{\Lie}{Lie}
\DeclareMathOperator{\Vect}{Vect}
\DeclareMathOperator{\Span}{Span}
%%%%%%%%%%%%%%%%%%%%%%%%%%%%%%%%%%%%%%%%%%%%%%%%%%%%%%%%%%%%%%%%%%%%%
\newcommand\restr[2]{{% we make the whole thing an ordinary symbol
  \left.\kern-\nulldelimiterspace % automatically resize the bar with \right
  #1 % the function
  \vphantom{\big|} % pretend it's a little taller at normal size
  \right|_{#2} % this is the delimiter
  }}
%%%%%%%%%%%%%%%%%%%%%%%%%%%%%%%%%%%%%%%%%%%%%%%%%%%%%%%%%%%%%%%%%%%%%

\newtheorem{theorem}{Theorem}[section]
\newtheorem*{theorem*}{Theorem}
\newtheorem{proposition}[theorem]{Proposition}
\newtheorem*{proposition*}{Proposition}
\newtheorem{conjecture}[theorem]{Conjecture}
\newtheorem*{conjecture*}{Conjecture}
\newtheorem{lemma}[theorem]{Lemma}
\newtheorem*{lemma*}{Lemma}
\newtheorem{corollary}[theorem]{Corollary}
\newtheorem*{corollary*}{Corollary}
\newtheorem{definition}[theorem]{Definition}
\newtheorem*{definition*}{Definition}
\newtheorem{example}[theorem]{Example}
\newtheorem*{example*}{Example}
\newtheorem{exercise}[theorem]{Exercise}
\newtheorem*{exercise*}{Exercise}
\newtheorem{remark}[theorem]{Remark}
\newtheorem*{remark*}{Remark}
\newtheorem{question}[theorem]{Question}
\newtheorem*{question*}{Question}
\newtheorem{claim}[theorem]{Claim}
\newtheorem*{claim*}{Claim}

\usepackage{tikz}
\tikzset{>=latex}
\usetikzlibrary{calc}
\usetikzlibrary{backgrounds}
\usetikzlibrary{patterns,decorations.pathreplacing}
\usetikzlibrary{spy}
\usepackage{pgfplots}
\pgfplotsset{compat=1.12}
\usepgfplotslibrary{colormaps}
\usepgfplotslibrary{patchplots}
\usepgfplotslibrary{fillbetween}
\usepackage{array} % for \newcolumntype macro
\newcolumntype{C}{>{{}}c<{{}}} % for columns with binary operators
\newcommand\vv{\multicolumn{1}{c}{\vdots}}

\newcommand{\rubric}[1]{{\color{blue}{#1}}}

\begin{document}
\title{MATH223 Homework 7\\ (due Friday, Nov/29, 11:59pm)}
\author{}
\date{}
\maketitle
\pagenumbering{gobble}

\begin{enumerate}
\item (This is Exercise 11 in Axler 5B) Suppose $V$ is a two-dimensional vector space, $T\in \mathcal{L}(V)$, and the matrix of $T$ with respect to some basis of $V$ is $
  \begin{pmatrix}
    a & b \\
    c & d
  \end{pmatrix}$.
  \begin{enumerate}[(a)]    
  \item (1 mark) Show that $T^2-(a+d)T+(ad-bc)I=0$.
  \item (1 mark) Show that the minimal polynomial of $T$ equals
    \[
      \begin{cases}
        z-a &\text{ if }b=c=0\text{ and }a=d, \\
        z^2-(a+d)z+(ad-bc) &\text{ otherwise}.
      \end{cases}
    \]
  \end{enumerate}
\item (2 marks, this is Exercise 16 in Axler 5B) Suppose $a_0,\ldots ,a_{n-1}\in \bF$. Let $T$ be the operator on $\bF^n$ whose matrix (with respect to the standard basis) is
  \[
    \begin{pmatrix}
      0 & & & & & -a_0 \\
      1 & 0 & & & & -a_1 \\
        & 1 & \ddots & & & -a_2 \\
        & & \ddots & & & \vdots \\
        & & & & 0 & -a_{n-2} \\
        & & & & 1 & -a_{n-1}
    \end{pmatrix}.
  \]
  Here all the entries of the matrix are $0$ except for all $1$'s on the line under the diagonal and the entries in the last column (some of which might also be $0$). Show that the minimal polynomial of $T$ is the polynomial
  \[
    a_0+a_1z+\ldots +a_{n-1}z^{n-1}+z^n.
  \]
\item (2 marks, this is Exercise 22 in Axler 5B) Suppose $V$ is finite-dimensional and $T\in \mathcal{L}(V)$. Prove that $T$ is invertible if and only if $I\in\operatorname{span}(T,T^2,\ldots ,T^{\dim V})$.
\item A \textbf{Pingala sequence} is a sequence $P_0,P_1,P_2,\ldots $ of real numbers satisfying
  \[
    P_n=P_{n-2}+P_{n-1}
  \]
  for $n\geq 2$.\footnote{You might notice that if $P_0=0,P_1=1$, then you get the famous Fibonacci sequence $(F_n)$. The Indian mathematician Pingala described this sequence about a \href{https://en.wikipedia.org/wiki/Fibonacci_sequence}{thousand years} before Fibonacci.}
  \begin{enumerate}[(a)]
  \item (1 mark) Show that the set of Pingala sequences is a subspace of $\bR^\bN$, the vector space of sequences of real numbers.
  \item (2 marks) Find a basis for the subspace of Pingala sequences and hence determine its dimension. Justify your answer.
  \item Define $T\in\mathcal{L}(\bR^2)$ by $T(x,y)=(y,x+y)$.
    \begin{enumerate}[(i)]
    \item (1 mark) Show that if $P_0,P_1,\ldots $ is a Pingala sequence, then $T^n(P_0,P_1)=(P_{n},P_{n+1})$ for $n\geq 0$.
    \item (1 mark) Find the eigenvalues of $T$.
    \item (2 mark) Find a basis of $\bR^2$ consisting of eigenvectors of $T$.
    \item (2 mark) Use the solution to the previous part to compute $T^n(0,1)$. Conclude that
      \[
        P_n=F_n=\frac{1}{\sqrt{5}}\left[ \left(\frac{1+\sqrt{5}}{2}\right)^n - \left(\frac{1-\sqrt{5}}{2}\right)^n\right]
      \]
      for each nonnegative integer $n$ (Not part of the question, just something to ponder: \textit{isn't this a funny way to write a number that is clearly an integer?}). 
    \item (1 mark) Use the previous part to conclude that if $n$ is a nonnegative integer, then the Fibonacci number $F_n$ is the integer that is closest to
      \[
        \frac{1}{\sqrt{5}}\left( \frac{1+\sqrt{5}}{2} \right)^n.
      \]
    \end{enumerate}
  \end{enumerate}
\item (This is Exercise 32 in Axler 3F) The \textbf{double dual space} of $V$, denoted $V^{\prime\prime}$, is defined to be the dual space of $V^\prime$, in other words, $V^{\prime\prime}=(V^{\prime})^{\prime}$. Define $\wedge:V\to V^{\prime\prime}$ by
  \[
    (\wedge(\vec v))(\varphi)=\varphi(\vec v)
  \]
  for each $v\in V$ and $\varphi\in V^\prime$.
  \begin{enumerate}[(a)]
  \item (1 mark) Show that $\wedge$ is a linear map from $V$ to $V^{\prime\prime}$.
  \item (2 marks) Show that if $T\in \mathcal{L}(V)$, then $T^{\prime\prime}\circ\wedge=\wedge\circ T$, where $T^{\prime\prime}=(T^\prime)^\prime$.
  \item (1 mark) Show that if $V$ is finite-dimensional, then $\wedge$ is an isomorphism from $V$ onto $V^{\prime\prime}$.
  \end{enumerate}
\end{enumerate}


\bibliography{c:/Dropbox/TeX/mybibliography}{}
% \bibliography{/media/bazse/storage/Dropbox/TeX/mybibliography}{}
\bibliographystyle{alpha}

\end{document}
