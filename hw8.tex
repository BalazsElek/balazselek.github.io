\documentclass[12pt]{article}
\usepackage{amsmath,amsthm,amssymb,color,enumerate}
\usepackage{graphicx}
\usepackage[backref=page]{hyperref}
\usepackage{cleveref}
\usepackage[all]{xy}
%\usepackage{ytableau}
%\usepackage{palatino}
\usepackage{mathpazo}
\usepackage{euler}
\usepackage{subcaption}

\hypersetup{colorlinks=true,linkcolor=blue,citecolor=red}
\voffset=-10mm %to print: dvips *.dvi; ps2pdf *.ps
\oddsidemargin=-20pt
\headheight=-20pt     \topmargin=0pt
%\textheight=658pt   \textwidth=495pt %return to this if possible
%\textheight=658pt   \textwidth=499.5pt
%textheight=659pt   \textwidth=499.5pt
\textheight=700pt   \textwidth=499.5pt
%parskip=.3pt plus 1pt
\parskip=.2pt plus .5pt
%%%%%%%%%%%%%%%%%%%%%%%%%%%%%%%%%%%%%%%%%%%%%%%%%%%%%%%%%%%%%%%%%%%%
\newcommand{\fg}{\mathfrak{g}}
\newcommand{\fh}{\mathfrak{h}}
\newcommand{\fb}{\mathfrak{b}}
\newcommand{\fn}{\mathfrak{n}}
\newcommand{\fm}{\mathfrak{m}}
\newcommand{\bx}{\mathbf{x}}
\newcommand{\bu}{\mathbf{u}}
\newcommand{\by}{\mathbf{y}}
\newcommand{\bz}{\mathbf{z}}
\newcommand{\bn}{\mathbf{n}}
\newcommand{\br}{\mathbf{r}}
\newcommand{\ba}{\mathbf{a}}
\newcommand{\bv}{\mathbf{v}}
\newcommand{\cA}{\mathcal{A}}
\newcommand{\cB}{\mathcal{B}}
\newcommand{\cC}{\mathcal{C}}
\newcommand{\cE}{\mathcal{E}}
\newcommand{\cF}{\mathcal{F}}
\newcommand{\cG}{\mathcal{G}}
\newcommand{\cH}{\mathcal{H}}
\newcommand{\cI}{\mathcal{I}}
\newcommand{\cK}{\mathcal{K}}
\newcommand{\cL}{\mathcal{L}}
\newcommand{\cN}{\mathcal{N}}
\newcommand{\cS}{\mathcal{S}}
\newcommand{\cT}{\mathcal{T}}
\newcommand{\bA}{\mathbb{A}}
\newcommand{\bC}{\mathbb{C}}
\newcommand{\bD}{\mathbb{D}}
\newcommand{\bR}{\mathbb{R}}
\newcommand{\bF}{\mathbb{F}}
\newcommand{\bN}{\mathbb{N}}
\newcommand{\bS}{\mathbb{S}}
\newcommand{\cO}{\mathcal{O}}
\newcommand{\fF}{\mathfrak{F}}
\newcommand{\bP}{\mathbb{P}}
\newcommand{\cY}{\mathcal{Y}}
\newcommand{\cZ}{\mathcal{Z}}
\newcommand{\bZ}{\mathbb{Z}}
\newcommand{\fZ}{\mathfrak{Z}}
\renewcommand{\sl}{\mathfrak{sl}}
\newcommand{\gl}{\mathfrak{gl}}
\newcommand{\Mat}{\mathrm{Mat}}
\newcommand{\sslash}{\mathbin{/\mkern-6mu/}}
\DeclareMathOperator{\Hom}{Hom}
\DeclareMathOperator{\Mor}{Mor}
\DeclareMathOperator{\End}{End}
\DeclareMathOperator{\Tor}{Tor}
\DeclareMathOperator{\Ext}{Ext}
\DeclareMathOperator{\Spec}{Spec}
\DeclareMathOperator{\Proj}{Proj}
\DeclareMathOperator{\Sym}{Sym}
\DeclareMathOperator{\rank}{rank}
\DeclareMathOperator{\supp}{supp}
\DeclareMathOperator{\Loc}{Loc}
\DeclareMathOperator{\Gr}{Gr}
\DeclareMathOperator{\coker}{coker}
\DeclareMathOperator{\im}{im}
\DeclareMathOperator{\ch}{ch}
\DeclareMathOperator{\ad}{ad}
\DeclareMathOperator{\rk}{rk}
\DeclareMathOperator{\gr}{gr}
\DeclareMathOperator{\wt}{wt}
\DeclareMathOperator{\Res}{Res}
\DeclareMathOperator{\Rep}{Rep}
\DeclareMathOperator{\Ind}{Ind}
\DeclareMathOperator{\Tr}{Tr}
\DeclareMathOperator{\Lie}{Lie}
\DeclareMathOperator{\Vect}{Vect}
\DeclareMathOperator{\Span}{Span}
%%%%%%%%%%%%%%%%%%%%%%%%%%%%%%%%%%%%%%%%%%%%%%%%%%%%%%%%%%%%%%%%%%%%%
\newcommand\restr[2]{{% we make the whole thing an ordinary symbol
  \left.\kern-\nulldelimiterspace % automatically resize the bar with \right
  #1 % the function
  \vphantom{\big|} % pretend it's a little taller at normal size
  \right|_{#2} % this is the delimiter
  }}
%%%%%%%%%%%%%%%%%%%%%%%%%%%%%%%%%%%%%%%%%%%%%%%%%%%%%%%%%%%%%%%%%%%%%

\newtheorem{theorem}{Theorem}[section]
\newtheorem*{theorem*}{Theorem}
\newtheorem{proposition}[theorem]{Proposition}
\newtheorem*{proposition*}{Proposition}
\newtheorem{conjecture}[theorem]{Conjecture}
\newtheorem*{conjecture*}{Conjecture}
\newtheorem{lemma}[theorem]{Lemma}
\newtheorem*{lemma*}{Lemma}
\newtheorem{corollary}[theorem]{Corollary}
\newtheorem*{corollary*}{Corollary}
\newtheorem{definition}[theorem]{Definition}
\newtheorem*{definition*}{Definition}
\newtheorem{example}[theorem]{Example}
\newtheorem*{example*}{Example}
\newtheorem{exercise}[theorem]{Exercise}
\newtheorem*{exercise*}{Exercise}
\newtheorem{remark}[theorem]{Remark}
\newtheorem*{remark*}{Remark}
\newtheorem{question}[theorem]{Question}
\newtheorem*{question*}{Question}
\newtheorem{claim}[theorem]{Claim}
\newtheorem*{claim*}{Claim}

% \usepackage{amsmath,amsthm,amssymb,color,enumerate}
\usepackage{graphicx}
\usepackage[backref=page]{hyperref}
\usepackage{cleveref}
\usepackage[all]{xy}
%\usepackage{ytableau}
%\usepackage{palatino}
\usepackage{mathpazo}
\usepackage{euler}
\usepackage{subcaption}

\hypersetup{colorlinks=true,linkcolor=blue,citecolor=red}
\voffset=-10mm %to print: dvips *.dvi; ps2pdf *.ps
\oddsidemargin=-20pt
\headheight=-20pt     \topmargin=0pt
%\textheight=658pt   \textwidth=495pt %return to this if possible
%\textheight=658pt   \textwidth=499.5pt
%textheight=659pt   \textwidth=499.5pt
\textheight=700pt   \textwidth=499.5pt
%parskip=.3pt plus 1pt
\parskip=.2pt plus .5pt
%%%%%%%%%%%%%%%%%%%%%%%%%%%%%%%%%%%%%%%%%%%%%%%%%%%%%%%%%%%%%%%%%%%%
\newcommand{\fg}{\mathfrak{g}}
\newcommand{\fh}{\mathfrak{h}}
\newcommand{\fb}{\mathfrak{b}}
\newcommand{\fn}{\mathfrak{n}}
\newcommand{\fm}{\mathfrak{m}}
\newcommand{\bx}{\mathbf{x}}
\newcommand{\bu}{\mathbf{u}}
\newcommand{\by}{\mathbf{y}}
\newcommand{\bz}{\mathbf{z}}
\newcommand{\bn}{\mathbf{n}}
\newcommand{\br}{\mathbf{r}}
\newcommand{\ba}{\mathbf{a}}
\newcommand{\bv}{\mathbf{v}}
\newcommand{\cA}{\mathcal{A}}
\newcommand{\cB}{\mathcal{B}}
\newcommand{\cC}{\mathcal{C}}
\newcommand{\cE}{\mathcal{E}}
\newcommand{\cF}{\mathcal{F}}
\newcommand{\cG}{\mathcal{G}}
\newcommand{\cH}{\mathcal{H}}
\newcommand{\cI}{\mathcal{I}}
\newcommand{\cK}{\mathcal{K}}
\newcommand{\cL}{\mathcal{L}}
\newcommand{\cN}{\mathcal{N}}
\newcommand{\cS}{\mathcal{S}}
\newcommand{\cT}{\mathcal{T}}
\newcommand{\bA}{\mathbb{A}}
\newcommand{\bC}{\mathbb{C}}
\newcommand{\bD}{\mathbb{D}}
\newcommand{\bR}{\mathbb{R}}
\newcommand{\bF}{\mathbb{F}}
\newcommand{\bN}{\mathbb{N}}
\newcommand{\bS}{\mathbb{S}}
\newcommand{\cO}{\mathcal{O}}
\newcommand{\fF}{\mathfrak{F}}
\newcommand{\bP}{\mathbb{P}}
\newcommand{\cY}{\mathcal{Y}}
\newcommand{\cZ}{\mathcal{Z}}
\newcommand{\bZ}{\mathbb{Z}}
\newcommand{\fZ}{\mathfrak{Z}}
\renewcommand{\sl}{\mathfrak{sl}}
\newcommand{\gl}{\mathfrak{gl}}
\newcommand{\Mat}{\mathrm{Mat}}
\newcommand{\sslash}{\mathbin{/\mkern-6mu/}}
\DeclareMathOperator{\Hom}{Hom}
\DeclareMathOperator{\Mor}{Mor}
\DeclareMathOperator{\End}{End}
\DeclareMathOperator{\Tor}{Tor}
\DeclareMathOperator{\Ext}{Ext}
\DeclareMathOperator{\Spec}{Spec}
\DeclareMathOperator{\Proj}{Proj}
\DeclareMathOperator{\Sym}{Sym}
\DeclareMathOperator{\rank}{rank}
\DeclareMathOperator{\supp}{supp}
\DeclareMathOperator{\Loc}{Loc}
\DeclareMathOperator{\Gr}{Gr}
\DeclareMathOperator{\coker}{coker}
\DeclareMathOperator{\im}{im}
\DeclareMathOperator{\ch}{ch}
\DeclareMathOperator{\ad}{ad}
\DeclareMathOperator{\rk}{rk}
\DeclareMathOperator{\gr}{gr}
\DeclareMathOperator{\wt}{wt}
\DeclareMathOperator{\Res}{Res}
\DeclareMathOperator{\Rep}{Rep}
\DeclareMathOperator{\Ind}{Ind}
\DeclareMathOperator{\Tr}{Tr}
\DeclareMathOperator{\Lie}{Lie}
\DeclareMathOperator{\Vect}{Vect}
\DeclareMathOperator{\Span}{Span}
%%%%%%%%%%%%%%%%%%%%%%%%%%%%%%%%%%%%%%%%%%%%%%%%%%%%%%%%%%%%%%%%%%%%%
\newcommand\restr[2]{{% we make the whole thing an ordinary symbol
  \left.\kern-\nulldelimiterspace % automatically resize the bar with \right
  #1 % the function
  \vphantom{\big|} % pretend it's a little taller at normal size
  \right|_{#2} % this is the delimiter
  }}
%%%%%%%%%%%%%%%%%%%%%%%%%%%%%%%%%%%%%%%%%%%%%%%%%%%%%%%%%%%%%%%%%%%%%

\newtheorem{theorem}{Theorem}[section]
\newtheorem*{theorem*}{Theorem}
\newtheorem{proposition}[theorem]{Proposition}
\newtheorem*{proposition*}{Proposition}
\newtheorem{conjecture}[theorem]{Conjecture}
\newtheorem*{conjecture*}{Conjecture}
\newtheorem{lemma}[theorem]{Lemma}
\newtheorem*{lemma*}{Lemma}
\newtheorem{corollary}[theorem]{Corollary}
\newtheorem*{corollary*}{Corollary}
\newtheorem{definition}[theorem]{Definition}
\newtheorem*{definition*}{Definition}
\newtheorem{example}[theorem]{Example}
\newtheorem*{example*}{Example}
\newtheorem{exercise}[theorem]{Exercise}
\newtheorem*{exercise*}{Exercise}
\newtheorem{remark}[theorem]{Remark}
\newtheorem*{remark*}{Remark}
\newtheorem{question}[theorem]{Question}
\newtheorem*{question*}{Question}
\newtheorem{claim}[theorem]{Claim}
\newtheorem*{claim*}{Claim}

\usepackage{tikz}
\tikzset{>=latex}
\usetikzlibrary{calc}
\usetikzlibrary{backgrounds}
\usetikzlibrary{patterns,decorations.pathreplacing}
\usetikzlibrary{spy}
\usepackage{pgfplots}
\pgfplotsset{compat=1.12}
\usepgfplotslibrary{colormaps}
\usepgfplotslibrary{patchplots}
\usepgfplotslibrary{fillbetween}
\usepackage{array} % for \newcolumntype macro
\newcolumntype{C}{>{{}}c<{{}}} % for columns with binary operators
\newcommand\vv{\multicolumn{1}{c}{\vdots}}

\newcommand{\rubric}[1]{{\color{blue}{#1}}}

\begin{document}
\title{MATH223 Homework 8\\ (due Friday, Dec/6, 11:59pm)}
\author{}
\date{}
\maketitle
\pagenumbering{gobble}

\begin{enumerate}
\item (2 marks) Let $A\in\bR^{m,n}$ be an $m\times n$ matrix with entries in $\bR$. We define the following:
  \begin{itemize}
  \item The \textbf{column space} $\operatorname{Col}(A)$ of $A$ is the span of the columns of $A$ (it is a subspace of $\bR^{m,1}$).
  \item The \textbf{row space} $\operatorname{Row}(A)$ of $A$ is the span of the rows of $A$ (it is a subspace of $\bR^{1,n}$).
  \item The \textbf{null space} $\operatorname{null}(A)$ of $A$ is $\{ X\in \bR^{n,1} \; |\; AX=0_{\bR^{m,1}} \}$.
  \end{itemize}
  Show that
  \[
    \left(\operatorname{Row}(A)^\perp\right)^t=\operatorname{null}(A)
  \]
  and
  \[
    \operatorname{Col}(A)^\perp=\operatorname{null}(A^t)
  \]
  where ${}^\perp$ is considered with respect to the dot product and $M^t$ denotes the transpose of the matrix $M$.
\item (3 marks) This exercise describes a practical way to compute the matrix of a projection onto a subspace. Let $V=\bR^{n,1}$ be the vector space of column vectors with $n$ components equipped with the standard dot product, and let $U\subseteq V$ be a subspace. Choose a basis $\vec u_1,\vec u_2,\ldots ,\vec u_m$ of $U$ and let
  \[
    A=( u_1 u_2 \ldots u_m)
  \]
  be the $n\times m$ matrix whose columns are the $u_j$-s. Let $\vec y\in V$. Prove that $\exists \vec x\in \bR^{m,1}$ such that
  \[
    A^tA\vec x =A^t\vec y
  \]
  and moreover, for any such $\vec x$, we have
  \[
    A\vec x=P_U\vec y,
  \]
  where $P_U$ is the orthogonal projection onto $U$.
\item (2 marks) Let $V=C[-1,1]$ be the vector space of continuous real-valued functions defined on the interval $[-1,1]$ with inner product $\langle f,g\rangle = \int_{-1}^1 f(t)g(t) \; dt$. Let $W_e$ and $W_o$ denote the subspaces of even and odd functions respectively. Prove that
  \[
    W_e=W_o^\perp.
  \]
  (\textbf{Hint:} We have proved in class that if $V$ is finite-dimensional inner product space, and $U$ is a subspace then $V=U\oplus U^\perp$. Since $V=C[-1,1]$ is infinite-dimensional, we could not use this theorem. However, it is true that $C[-1,1]=W_e\oplus W_e^\perp$ and $C[-1,1]=W_o\oplus W_o^\perp$. This should help you in answering this question. Before you jump to conclusions, note that if $V=U\oplus W$ and $V=U\oplus\widetilde{W}$, we can not in general conclude that $W=\widetilde{W}$.)
\item (2 marks, this is Exercise 6 in Axler 6C) Suppose $V$ is finite-dimensional and $T\in \mathcal{L}(V,W)$.
  \[
    T=TP_{(\operatorname{null}T)^\perp}=P_{\operatorname{range} T}T.
  \]
\item (2 marks) Let $M\in \bC^{n,n}$ be an $n\times n$ matrix with complex entries. Let $M^*$ denote the \textbf{conjugate transpose} of $M$, that is $M^*$ is an $n\times n$ complex matrix whose entries are
  \[
    M^*_{(i,j)}=\overline{M_{(j,i)}}.
  \]
  Prove that
  \[
    MM^*=I_{\bC^n}
  \]
  if and only if the rows of $M$ form an orthonormal basis for $\bC^n$.
\item (2 marks, this is Exercise 17 in Axler 8A) Suppose $T\in \mathcal{L}(V)$ is nilpotent and $m$ is a positive integer such that $T^m=0$.
  \begin{enumerate}[(a)]
  \item Prove that $I-T$ is invertible and that $(I-T)^{-1}=I+T+\ldots +T^{m-1}$.
  \item Explain how you would guess the formula above.
  \end{enumerate}
\item (2 marks, this is Exercise 3 in Axler 8B) Suppose $T\in \mathcal{L}(V)$. Suppose $S\in \mathcal{L}(V)$ is invertible. Prove that $T$ and $S^{-1}TS$ have the same eigenvalues with the same multiplicities.
\item (2 marks, this is Exercise 9 in Axler 8B) Suppose $\bF=\bC$ and $T\in \mathcal{L}(V)$. Prove that there exist $D,N\in \mathcal{L}(V)$ such that $T=D+N$, the operator $D$ is diagonalizable, $N$ is nilpotent, and $DN=ND$.
\item (3 marks, this is Exercise 14 in Axler 8C) Suppose $\bF=\bC$ and $T\in \mathcal{L}(V)$. Prove that there does not exist a direct sum decomposition of $V$ into two nonzero subspaces invariant under $T$ if and only if the minimal polynomial of $T$ is of the form $(z-\lambda)^{\dim V}$ for some $\lambda\in \bC$.
\end{enumerate}


\bibliography{c:/Dropbox/TeX/mybibliography}{}
% \bibliography{/media/bazse/storage/Dropbox/TeX/mybibliography}{}
\bibliographystyle{alpha}

\end{document}
