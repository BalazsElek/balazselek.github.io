\documentclass[12pt]{article}
%\usepackage{amsmath,amsthm,amssymb,color,enumerate}
\usepackage{graphicx}
\usepackage[backref=page]{hyperref}
\usepackage{cleveref}
\usepackage[all]{xy}
%\usepackage{ytableau}
%\usepackage{palatino}
\usepackage{mathpazo}
\usepackage{euler}
\usepackage{subcaption}

\hypersetup{colorlinks=true,linkcolor=blue,citecolor=red}
\voffset=-10mm %to print: dvips *.dvi; ps2pdf *.ps
\oddsidemargin=-20pt
\headheight=-20pt     \topmargin=0pt
%\textheight=658pt   \textwidth=495pt %return to this if possible
%\textheight=658pt   \textwidth=499.5pt
%textheight=659pt   \textwidth=499.5pt
\textheight=700pt   \textwidth=499.5pt
%parskip=.3pt plus 1pt
\parskip=.2pt plus .5pt
%%%%%%%%%%%%%%%%%%%%%%%%%%%%%%%%%%%%%%%%%%%%%%%%%%%%%%%%%%%%%%%%%%%%
\newcommand{\fg}{\mathfrak{g}}
\newcommand{\fh}{\mathfrak{h}}
\newcommand{\fb}{\mathfrak{b}}
\newcommand{\fn}{\mathfrak{n}}
\newcommand{\fm}{\mathfrak{m}}
\newcommand{\bx}{\mathbf{x}}
\newcommand{\bu}{\mathbf{u}}
\newcommand{\by}{\mathbf{y}}
\newcommand{\bz}{\mathbf{z}}
\newcommand{\bn}{\mathbf{n}}
\newcommand{\br}{\mathbf{r}}
\newcommand{\ba}{\mathbf{a}}
\newcommand{\bv}{\mathbf{v}}
\newcommand{\cA}{\mathcal{A}}
\newcommand{\cB}{\mathcal{B}}
\newcommand{\cC}{\mathcal{C}}
\newcommand{\cE}{\mathcal{E}}
\newcommand{\cF}{\mathcal{F}}
\newcommand{\cG}{\mathcal{G}}
\newcommand{\cH}{\mathcal{H}}
\newcommand{\cI}{\mathcal{I}}
\newcommand{\cK}{\mathcal{K}}
\newcommand{\cL}{\mathcal{L}}
\newcommand{\cN}{\mathcal{N}}
\newcommand{\cS}{\mathcal{S}}
\newcommand{\cT}{\mathcal{T}}
\newcommand{\bA}{\mathbb{A}}
\newcommand{\bC}{\mathbb{C}}
\newcommand{\bD}{\mathbb{D}}
\newcommand{\bR}{\mathbb{R}}
\newcommand{\bF}{\mathbb{F}}
\newcommand{\bN}{\mathbb{N}}
\newcommand{\bS}{\mathbb{S}}
\newcommand{\cO}{\mathcal{O}}
\newcommand{\fF}{\mathfrak{F}}
\newcommand{\bP}{\mathbb{P}}
\newcommand{\cY}{\mathcal{Y}}
\newcommand{\cZ}{\mathcal{Z}}
\newcommand{\bZ}{\mathbb{Z}}
\newcommand{\fZ}{\mathfrak{Z}}
\renewcommand{\sl}{\mathfrak{sl}}
\newcommand{\gl}{\mathfrak{gl}}
\newcommand{\Mat}{\mathrm{Mat}}
\newcommand{\sslash}{\mathbin{/\mkern-6mu/}}
\DeclareMathOperator{\Hom}{Hom}
\DeclareMathOperator{\Mor}{Mor}
\DeclareMathOperator{\End}{End}
\DeclareMathOperator{\Tor}{Tor}
\DeclareMathOperator{\Ext}{Ext}
\DeclareMathOperator{\Spec}{Spec}
\DeclareMathOperator{\Proj}{Proj}
\DeclareMathOperator{\Sym}{Sym}
\DeclareMathOperator{\rank}{rank}
\DeclareMathOperator{\supp}{supp}
\DeclareMathOperator{\Loc}{Loc}
\DeclareMathOperator{\Gr}{Gr}
\DeclareMathOperator{\coker}{coker}
\DeclareMathOperator{\im}{im}
\DeclareMathOperator{\ch}{ch}
\DeclareMathOperator{\ad}{ad}
\DeclareMathOperator{\rk}{rk}
\DeclareMathOperator{\gr}{gr}
\DeclareMathOperator{\wt}{wt}
\DeclareMathOperator{\Res}{Res}
\DeclareMathOperator{\Rep}{Rep}
\DeclareMathOperator{\Ind}{Ind}
\DeclareMathOperator{\Tr}{Tr}
\DeclareMathOperator{\Lie}{Lie}
\DeclareMathOperator{\Vect}{Vect}
\DeclareMathOperator{\Span}{Span}
%%%%%%%%%%%%%%%%%%%%%%%%%%%%%%%%%%%%%%%%%%%%%%%%%%%%%%%%%%%%%%%%%%%%%
\newcommand\restr[2]{{% we make the whole thing an ordinary symbol
  \left.\kern-\nulldelimiterspace % automatically resize the bar with \right
  #1 % the function
  \vphantom{\big|} % pretend it's a little taller at normal size
  \right|_{#2} % this is the delimiter
  }}
%%%%%%%%%%%%%%%%%%%%%%%%%%%%%%%%%%%%%%%%%%%%%%%%%%%%%%%%%%%%%%%%%%%%%

\newtheorem{theorem}{Theorem}[section]
\newtheorem*{theorem*}{Theorem}
\newtheorem{proposition}[theorem]{Proposition}
\newtheorem*{proposition*}{Proposition}
\newtheorem{conjecture}[theorem]{Conjecture}
\newtheorem*{conjecture*}{Conjecture}
\newtheorem{lemma}[theorem]{Lemma}
\newtheorem*{lemma*}{Lemma}
\newtheorem{corollary}[theorem]{Corollary}
\newtheorem*{corollary*}{Corollary}
\newtheorem{definition}[theorem]{Definition}
\newtheorem*{definition*}{Definition}
\newtheorem{example}[theorem]{Example}
\newtheorem*{example*}{Example}
\newtheorem{exercise}[theorem]{Exercise}
\newtheorem*{exercise*}{Exercise}
\newtheorem{remark}[theorem]{Remark}
\newtheorem*{remark*}{Remark}
\newtheorem{question}[theorem]{Question}
\newtheorem*{question*}{Question}
\newtheorem{claim}[theorem]{Claim}
\newtheorem*{claim*}{Claim}

\usepackage{amsmath,amsthm,amssymb,color,enumerate}
\usepackage{graphicx}
\usepackage[backref=page]{hyperref}
\usepackage{cleveref}
\usepackage[all]{xy}
%\usepackage{ytableau}
%\usepackage{palatino}
\usepackage{mathpazo}
\usepackage{euler}
\usepackage{subcaption}

\hypersetup{colorlinks=true,linkcolor=blue,citecolor=red}
\voffset=-10mm %to print: dvips *.dvi; ps2pdf *.ps
\oddsidemargin=-20pt
\headheight=-20pt     \topmargin=0pt
%\textheight=658pt   \textwidth=495pt %return to this if possible
%\textheight=658pt   \textwidth=499.5pt
%textheight=659pt   \textwidth=499.5pt
\textheight=700pt   \textwidth=499.5pt
%parskip=.3pt plus 1pt
\parskip=.2pt plus .5pt
%%%%%%%%%%%%%%%%%%%%%%%%%%%%%%%%%%%%%%%%%%%%%%%%%%%%%%%%%%%%%%%%%%%%
\newcommand{\fg}{\mathfrak{g}}
\newcommand{\fh}{\mathfrak{h}}
\newcommand{\fb}{\mathfrak{b}}
\newcommand{\fn}{\mathfrak{n}}
\newcommand{\fm}{\mathfrak{m}}
\newcommand{\bx}{\mathbf{x}}
\newcommand{\bu}{\mathbf{u}}
\newcommand{\by}{\mathbf{y}}
\newcommand{\bz}{\mathbf{z}}
\newcommand{\bn}{\mathbf{n}}
\newcommand{\br}{\mathbf{r}}
\newcommand{\ba}{\mathbf{a}}
\newcommand{\bv}{\mathbf{v}}
\newcommand{\cA}{\mathcal{A}}
\newcommand{\cB}{\mathcal{B}}
\newcommand{\cC}{\mathcal{C}}
\newcommand{\cE}{\mathcal{E}}
\newcommand{\cF}{\mathcal{F}}
\newcommand{\cG}{\mathcal{G}}
\newcommand{\cH}{\mathcal{H}}
\newcommand{\cI}{\mathcal{I}}
\newcommand{\cK}{\mathcal{K}}
\newcommand{\cL}{\mathcal{L}}
\newcommand{\cN}{\mathcal{N}}
\newcommand{\cS}{\mathcal{S}}
\newcommand{\cT}{\mathcal{T}}
\newcommand{\bA}{\mathbb{A}}
\newcommand{\bC}{\mathbb{C}}
\newcommand{\bD}{\mathbb{D}}
\newcommand{\bR}{\mathbb{R}}
\newcommand{\bF}{\mathbb{F}}
\newcommand{\bN}{\mathbb{N}}
\newcommand{\bS}{\mathbb{S}}
\newcommand{\cO}{\mathcal{O}}
\newcommand{\fF}{\mathfrak{F}}
\newcommand{\bP}{\mathbb{P}}
\newcommand{\cY}{\mathcal{Y}}
\newcommand{\cZ}{\mathcal{Z}}
\newcommand{\bZ}{\mathbb{Z}}
\newcommand{\fZ}{\mathfrak{Z}}
\renewcommand{\sl}{\mathfrak{sl}}
\newcommand{\gl}{\mathfrak{gl}}
\newcommand{\Mat}{\mathrm{Mat}}
\newcommand{\sslash}{\mathbin{/\mkern-6mu/}}
\DeclareMathOperator{\Hom}{Hom}
\DeclareMathOperator{\Mor}{Mor}
\DeclareMathOperator{\End}{End}
\DeclareMathOperator{\Tor}{Tor}
\DeclareMathOperator{\Ext}{Ext}
\DeclareMathOperator{\Spec}{Spec}
\DeclareMathOperator{\Proj}{Proj}
\DeclareMathOperator{\Sym}{Sym}
\DeclareMathOperator{\rank}{rank}
\DeclareMathOperator{\supp}{supp}
\DeclareMathOperator{\Loc}{Loc}
\DeclareMathOperator{\Gr}{Gr}
\DeclareMathOperator{\coker}{coker}
\DeclareMathOperator{\im}{im}
\DeclareMathOperator{\ch}{ch}
\DeclareMathOperator{\ad}{ad}
\DeclareMathOperator{\rk}{rk}
\DeclareMathOperator{\gr}{gr}
\DeclareMathOperator{\wt}{wt}
\DeclareMathOperator{\Res}{Res}
\DeclareMathOperator{\Rep}{Rep}
\DeclareMathOperator{\Ind}{Ind}
\DeclareMathOperator{\Tr}{Tr}
\DeclareMathOperator{\Lie}{Lie}
\DeclareMathOperator{\Vect}{Vect}
\DeclareMathOperator{\Span}{Span}
%%%%%%%%%%%%%%%%%%%%%%%%%%%%%%%%%%%%%%%%%%%%%%%%%%%%%%%%%%%%%%%%%%%%%
\newcommand\restr[2]{{% we make the whole thing an ordinary symbol
  \left.\kern-\nulldelimiterspace % automatically resize the bar with \right
  #1 % the function
  \vphantom{\big|} % pretend it's a little taller at normal size
  \right|_{#2} % this is the delimiter
  }}
%%%%%%%%%%%%%%%%%%%%%%%%%%%%%%%%%%%%%%%%%%%%%%%%%%%%%%%%%%%%%%%%%%%%%

\newtheorem{theorem}{Theorem}[section]
\newtheorem*{theorem*}{Theorem}
\newtheorem{proposition}[theorem]{Proposition}
\newtheorem*{proposition*}{Proposition}
\newtheorem{conjecture}[theorem]{Conjecture}
\newtheorem*{conjecture*}{Conjecture}
\newtheorem{lemma}[theorem]{Lemma}
\newtheorem*{lemma*}{Lemma}
\newtheorem{corollary}[theorem]{Corollary}
\newtheorem*{corollary*}{Corollary}
\newtheorem{definition}[theorem]{Definition}
\newtheorem*{definition*}{Definition}
\newtheorem{example}[theorem]{Example}
\newtheorem*{example*}{Example}
\newtheorem{exercise}[theorem]{Exercise}
\newtheorem*{exercise*}{Exercise}
\newtheorem{remark}[theorem]{Remark}
\newtheorem*{remark*}{Remark}
\newtheorem{question}[theorem]{Question}
\newtheorem*{question*}{Question}
\newtheorem{claim}[theorem]{Claim}
\newtheorem*{claim*}{Claim}


\usepackage{array} % for \newcolumntype macro
\newcolumntype{C}{>{{}}c<{{}}} % for columns with binary operators
\newcommand\vv{\multicolumn{1}{c}{\vdots}}

\begin{document}
\title{MATH223 Homework 2\\ (due Friday, September 19, 11:59pm)}
\author{}
\date{}
\maketitle
\pagenumbering{gobble}

\begin{enumerate}
\item As mentioned in the ``Digression on fields'' section in Axler chapter 1A, a \textbf{field} is a set with at least two elements, $0\neq 1$ with two operations, addition ($+$) and multiplication $(\cdot)$ satisfying the ``Properties of complex arithmetic'' (listed in 1.3).
  \begin{enumerate}
  \item (1 mark) Consider the set $\bF_2:=\{0,1\}$ with the following operations:
    \[
      \begin{array}[h]{c | c c}
        + & 0 & 1 \\
        \hline
        0 & 0 & 1 \\
        1 & 1 & 0
      \end{array}
      \qquad
      \begin{array}[h]{c | c c}
        \cdot & 0 & 1 \\
        \hline
        0 & 0 & 0 \\
        1 & 0 & 1
      \end{array}
    \]

    Verify that $\bF_2$ is a field.
  \item (1 mark) Consider the set $\bF_4:=\{ 0,1,x, y\}$ with the following operations:
    \[
      \begin{array}[h]{c | c c c c }
        + & 0 & 1 & x & y \\
        \hline
        0 & 0 & 1 & x & y \\
        1 & 1 & 0 & y & x \\
        x & x & y & 0 & 1 \\
        y & y & x & 1 & 0
      \end{array}
      \qquad
      \begin{array}[h]{c | c c c c }
        \cdot & 0 & 1 & x & y \\
        \hline
        0 & 0 & 0 & 0 & 0 \\
        1 & 0 & 1 & x & y \\
        x & 0 & x & y & 1 \\
        y & 0 & y & 1 & x
      \end{array}      
    \]
    Verify that $\bF_4$ is a field. (\textbf{Note:} Verifying associativity would take a lot of writing, so you can just check one (nontrivial) case by hand and state that the others are similar. You can do the same for commutativity and distributivity)
  \item (1 mark) We can think of $\bF_2$ as ``integers modulo $2$''. That is, we consider two integers $m$ and $n$ equivalent if they have the same remainder modulo $2$ (i.e. if they are both odd, or are both even). Then the usual addition and multiplication of integers define the operations $+$ and $\cdot$\footnote{this can be made precise using the concept of an equivalence relation and quotient rings}. The field $\bF_4$ has $4$ elements, can you think of $\bF_4$ as ``integers modulo $4$''? Explain your answer.
  \end{enumerate}
\item (2 marks, this is Exercise 12 in Axler 1C) Prove that the union of two subpspaces of $V$ is a subspace if and only if one of the subspaces is contained in the other.
\item (1 mark) Find a vector space $V$ where the union of three subspaces is a subspace, but none of the three subspaces contains the other two. (\textbf{Hint:} This will only work if $\bF=\bF_2$)
\item (3 marks, this is Exercise 13 in Axler 1C) Prove that if $V$ is a vector space over a field $\bF$ where $1+1\neq 0$, then the union of three subspaces of $V$ is a subspace if and only if one of the subspaces contains the other two.
\item Just like $\bF_2$, by considering integers modulo $3$, we can define a structure of a field on the set $\{0,1,2\}$. In this question, we will study geometry in the vector space $\bF_3^4$.
  \begin{enumerate}
  \item (1 mark) How many points does $\bF_3^4$ have?
  \item (1 mark) Recall from geometry that a line in $\bR^n$ can be described as the set of points:
    \[
      \left\{ \vec v+t\vec w \; | \;  t\in \bR \right\}
    \]
    for some fixed vectors $\vec v, \vec w\in \bR^n, \vec w \neq \vec 0$.
    
    Analogously, we define a line in $\bF_3^4$ as the set of points
    \[
      \left\{ \vec v+t\vec w \; | \;  t\in \bF_3 \right\}
    \]
    for some fixed vectors $\vec v, \vec w\in \bF_3^n, \vec w \neq \vec 0$.
    
    How many points are there on a line in $\bF_3^4$?
  \item (2 mark) Prove the following: three distinct points $\vec v_1, \vec v_2,\vec v_3\in \bF_3^4$ are on a line in $\bF_3^4$ if and only if $\vec v_1 + \vec v_2 + \vec v_3=\vec 0$.
  \item (1 mark) Which lines in $\bF_3^4$ are subspaces?
  \end{enumerate}
\item The board game SET is played with the set of cards on \Cref{fig:set}.
  \begin{figure}[h]
    \centering
    \includegraphics[width=0.6\textwidth]{set_cards.png}
    \caption{The cards in SET}
    \label{fig:set}
  \end{figure}
  The image on each card has four features:
  \begin{itemize}
  \item Color: red, green or purple,
  \item Number: one, two or three symbols,
  \item Shape: diamond, oval or squiggle,
  \item Shading: open, striped or solid.
  \end{itemize}
  Three cards are said to constitute a SET if, for each of the features, the cards display the feature as either all the same, or all different. For example, the two sets of $3$ boxed cards on \Cref{fig:sets} consistute SETs.
  \begin{figure}[h]
    \centering
    \begin{subfigure}[b]{0.35\textwidth}
      \includegraphics[width=\textwidth]{set1.png}
      \caption{A SET, where three features are the same and one is different}
      \label{fig:set1}
    \end{subfigure}
    \qquad
    \begin{subfigure}[b]{0.35\textwidth}
      \centering
      \includegraphics[width=\textwidth]{set2.png}
      \caption{A SET where all four features are different}
      \label{fig:set2}
    \end{subfigure}
    \caption{SETs}
    \label{fig:sets}
  \end{figure}  
  \begin{enumerate}
  \item (1 mark) Find a bijection from the cards in SET to $\bF_3^4$ (\textbf{Hint:} of course there are many bijections between sets of the same size, you should think of one that seems ``natural'', so, for example, it should not take too long to describe). 
  \item (2 marks) Find a condition on triples of (distinct) points in $\bF_3^4$ that is equivalent to (using your bijection) the three corresponding cards constituting a SET. Explain your answer carefully. (\textbf{Hint:} look for a linear equation)
  \item (1 mark) Reinterpret the condition you found in the previous part in terms of geometry (\textbf{Hint:} use Question 5).
  \item (2 marks) The \href{https://en.wikipedia.org/wiki/Set_(card_game)}{Wikipedia page} for SET makes the following claim:
    \begin{center}
      \textit{Given any two cards from the deck, there is one and only one other card that forms a set with them.}
    \end{center}
    Prove this claim.
  \item (no marks) Next time you see people play this game, you can tell them:
    \begin{center}
    \textit{Oh, I see, you are looking for \underline{\phantom{lines}} in a \underline{\phantom{vector space}} over the field \underline{\phantom{$\bF_3^4$}} }.\footnote{Despite the playful appearance of this question, there are some \href{https://terrytao.wordpress.com/2007/02/23/open-question-best-bounds-for-cap-sets/}{pretty serious} open problems closely related to this!}
    \end{center}
    
  \end{enumerate}
\end{enumerate}


\bibliography{c:/Dropbox/TeX/mybibliography}{}
% \bibliography{/media/bazse/storage/Dropbox/TeX/mybibliography}{}
\bibliographystyle{alpha}

\end{document}