\documentclass[12pt]{article}
\usepackage{amsmath,amsthm,amssymb,color,enumerate}
\usepackage{graphicx}
\usepackage[backref=page]{hyperref}
\usepackage{cleveref}
\usepackage[all]{xy}
%\usepackage{ytableau}
%\usepackage{palatino}
\usepackage{mathpazo}
\usepackage{euler}
\usepackage{subcaption}

\hypersetup{colorlinks=true,linkcolor=blue,citecolor=red}
\voffset=-10mm %to print: dvips *.dvi; ps2pdf *.ps
\oddsidemargin=-20pt
\headheight=-20pt     \topmargin=0pt
%\textheight=658pt   \textwidth=495pt %return to this if possible
%\textheight=658pt   \textwidth=499.5pt
%textheight=659pt   \textwidth=499.5pt
\textheight=700pt   \textwidth=499.5pt
%parskip=.3pt plus 1pt
\parskip=.2pt plus .5pt
%%%%%%%%%%%%%%%%%%%%%%%%%%%%%%%%%%%%%%%%%%%%%%%%%%%%%%%%%%%%%%%%%%%%
\newcommand{\fg}{\mathfrak{g}}
\newcommand{\fh}{\mathfrak{h}}
\newcommand{\fb}{\mathfrak{b}}
\newcommand{\fn}{\mathfrak{n}}
\newcommand{\fm}{\mathfrak{m}}
\newcommand{\bx}{\mathbf{x}}
\newcommand{\bu}{\mathbf{u}}
\newcommand{\by}{\mathbf{y}}
\newcommand{\bz}{\mathbf{z}}
\newcommand{\bn}{\mathbf{n}}
\newcommand{\br}{\mathbf{r}}
\newcommand{\ba}{\mathbf{a}}
\newcommand{\bv}{\mathbf{v}}
\newcommand{\cA}{\mathcal{A}}
\newcommand{\cB}{\mathcal{B}}
\newcommand{\cC}{\mathcal{C}}
\newcommand{\cE}{\mathcal{E}}
\newcommand{\cF}{\mathcal{F}}
\newcommand{\cG}{\mathcal{G}}
\newcommand{\cH}{\mathcal{H}}
\newcommand{\cI}{\mathcal{I}}
\newcommand{\cK}{\mathcal{K}}
\newcommand{\cL}{\mathcal{L}}
\newcommand{\cN}{\mathcal{N}}
\newcommand{\cS}{\mathcal{S}}
\newcommand{\cT}{\mathcal{T}}
\newcommand{\bA}{\mathbb{A}}
\newcommand{\bC}{\mathbb{C}}
\newcommand{\bD}{\mathbb{D}}
\newcommand{\bR}{\mathbb{R}}
\newcommand{\bF}{\mathbb{F}}
\newcommand{\bN}{\mathbb{N}}
\newcommand{\bS}{\mathbb{S}}
\newcommand{\cO}{\mathcal{O}}
\newcommand{\fF}{\mathfrak{F}}
\newcommand{\bP}{\mathbb{P}}
\newcommand{\cY}{\mathcal{Y}}
\newcommand{\cZ}{\mathcal{Z}}
\newcommand{\bZ}{\mathbb{Z}}
\newcommand{\fZ}{\mathfrak{Z}}
\renewcommand{\sl}{\mathfrak{sl}}
\newcommand{\gl}{\mathfrak{gl}}
\newcommand{\Mat}{\mathrm{Mat}}
\newcommand{\sslash}{\mathbin{/\mkern-6mu/}}
\DeclareMathOperator{\Hom}{Hom}
\DeclareMathOperator{\Mor}{Mor}
\DeclareMathOperator{\End}{End}
\DeclareMathOperator{\Tor}{Tor}
\DeclareMathOperator{\Ext}{Ext}
\DeclareMathOperator{\Spec}{Spec}
\DeclareMathOperator{\Proj}{Proj}
\DeclareMathOperator{\Sym}{Sym}
\DeclareMathOperator{\rank}{rank}
\DeclareMathOperator{\supp}{supp}
\DeclareMathOperator{\Loc}{Loc}
\DeclareMathOperator{\Gr}{Gr}
\DeclareMathOperator{\coker}{coker}
\DeclareMathOperator{\im}{im}
\DeclareMathOperator{\ch}{ch}
\DeclareMathOperator{\ad}{ad}
\DeclareMathOperator{\rk}{rk}
\DeclareMathOperator{\gr}{gr}
\DeclareMathOperator{\wt}{wt}
\DeclareMathOperator{\Res}{Res}
\DeclareMathOperator{\Rep}{Rep}
\DeclareMathOperator{\Ind}{Ind}
\DeclareMathOperator{\Tr}{Tr}
\DeclareMathOperator{\Lie}{Lie}
\DeclareMathOperator{\Vect}{Vect}
\DeclareMathOperator{\Span}{Span}
%%%%%%%%%%%%%%%%%%%%%%%%%%%%%%%%%%%%%%%%%%%%%%%%%%%%%%%%%%%%%%%%%%%%%
\newcommand\restr[2]{{% we make the whole thing an ordinary symbol
  \left.\kern-\nulldelimiterspace % automatically resize the bar with \right
  #1 % the function
  \vphantom{\big|} % pretend it's a little taller at normal size
  \right|_{#2} % this is the delimiter
  }}
%%%%%%%%%%%%%%%%%%%%%%%%%%%%%%%%%%%%%%%%%%%%%%%%%%%%%%%%%%%%%%%%%%%%%

\newtheorem{theorem}{Theorem}[section]
\newtheorem*{theorem*}{Theorem}
\newtheorem{proposition}[theorem]{Proposition}
\newtheorem*{proposition*}{Proposition}
\newtheorem{conjecture}[theorem]{Conjecture}
\newtheorem*{conjecture*}{Conjecture}
\newtheorem{lemma}[theorem]{Lemma}
\newtheorem*{lemma*}{Lemma}
\newtheorem{corollary}[theorem]{Corollary}
\newtheorem*{corollary*}{Corollary}
\newtheorem{definition}[theorem]{Definition}
\newtheorem*{definition*}{Definition}
\newtheorem{example}[theorem]{Example}
\newtheorem*{example*}{Example}
\newtheorem{exercise}[theorem]{Exercise}
\newtheorem*{exercise*}{Exercise}
\newtheorem{remark}[theorem]{Remark}
\newtheorem*{remark*}{Remark}
\newtheorem{question}[theorem]{Question}
\newtheorem*{question*}{Question}
\newtheorem{claim}[theorem]{Claim}
\newtheorem*{claim*}{Claim}

%\usepackage{amsmath,amsthm,amssymb,color,enumerate}
\usepackage{graphicx}
\usepackage[backref=page]{hyperref}
\usepackage{cleveref}
\usepackage[all]{xy}
%\usepackage{ytableau}
%\usepackage{palatino}
\usepackage{mathpazo}
\usepackage{euler}
\usepackage{subcaption}

\hypersetup{colorlinks=true,linkcolor=blue,citecolor=red}
\voffset=-10mm %to print: dvips *.dvi; ps2pdf *.ps
\oddsidemargin=-20pt
\headheight=-20pt     \topmargin=0pt
%\textheight=658pt   \textwidth=495pt %return to this if possible
%\textheight=658pt   \textwidth=499.5pt
%textheight=659pt   \textwidth=499.5pt
\textheight=700pt   \textwidth=499.5pt
%parskip=.3pt plus 1pt
\parskip=.2pt plus .5pt
%%%%%%%%%%%%%%%%%%%%%%%%%%%%%%%%%%%%%%%%%%%%%%%%%%%%%%%%%%%%%%%%%%%%
\newcommand{\fg}{\mathfrak{g}}
\newcommand{\fh}{\mathfrak{h}}
\newcommand{\fb}{\mathfrak{b}}
\newcommand{\fn}{\mathfrak{n}}
\newcommand{\fm}{\mathfrak{m}}
\newcommand{\bx}{\mathbf{x}}
\newcommand{\bu}{\mathbf{u}}
\newcommand{\by}{\mathbf{y}}
\newcommand{\bz}{\mathbf{z}}
\newcommand{\bn}{\mathbf{n}}
\newcommand{\br}{\mathbf{r}}
\newcommand{\ba}{\mathbf{a}}
\newcommand{\bv}{\mathbf{v}}
\newcommand{\cA}{\mathcal{A}}
\newcommand{\cB}{\mathcal{B}}
\newcommand{\cC}{\mathcal{C}}
\newcommand{\cE}{\mathcal{E}}
\newcommand{\cF}{\mathcal{F}}
\newcommand{\cG}{\mathcal{G}}
\newcommand{\cH}{\mathcal{H}}
\newcommand{\cI}{\mathcal{I}}
\newcommand{\cK}{\mathcal{K}}
\newcommand{\cL}{\mathcal{L}}
\newcommand{\cN}{\mathcal{N}}
\newcommand{\cS}{\mathcal{S}}
\newcommand{\cT}{\mathcal{T}}
\newcommand{\bA}{\mathbb{A}}
\newcommand{\bC}{\mathbb{C}}
\newcommand{\bD}{\mathbb{D}}
\newcommand{\bR}{\mathbb{R}}
\newcommand{\bF}{\mathbb{F}}
\newcommand{\bN}{\mathbb{N}}
\newcommand{\bS}{\mathbb{S}}
\newcommand{\cO}{\mathcal{O}}
\newcommand{\fF}{\mathfrak{F}}
\newcommand{\bP}{\mathbb{P}}
\newcommand{\cY}{\mathcal{Y}}
\newcommand{\cZ}{\mathcal{Z}}
\newcommand{\bZ}{\mathbb{Z}}
\newcommand{\fZ}{\mathfrak{Z}}
\renewcommand{\sl}{\mathfrak{sl}}
\newcommand{\gl}{\mathfrak{gl}}
\newcommand{\Mat}{\mathrm{Mat}}
\newcommand{\sslash}{\mathbin{/\mkern-6mu/}}
\DeclareMathOperator{\Hom}{Hom}
\DeclareMathOperator{\Mor}{Mor}
\DeclareMathOperator{\End}{End}
\DeclareMathOperator{\Tor}{Tor}
\DeclareMathOperator{\Ext}{Ext}
\DeclareMathOperator{\Spec}{Spec}
\DeclareMathOperator{\Proj}{Proj}
\DeclareMathOperator{\Sym}{Sym}
\DeclareMathOperator{\rank}{rank}
\DeclareMathOperator{\supp}{supp}
\DeclareMathOperator{\Loc}{Loc}
\DeclareMathOperator{\Gr}{Gr}
\DeclareMathOperator{\coker}{coker}
\DeclareMathOperator{\im}{im}
\DeclareMathOperator{\ch}{ch}
\DeclareMathOperator{\ad}{ad}
\DeclareMathOperator{\rk}{rk}
\DeclareMathOperator{\gr}{gr}
\DeclareMathOperator{\wt}{wt}
\DeclareMathOperator{\Res}{Res}
\DeclareMathOperator{\Rep}{Rep}
\DeclareMathOperator{\Ind}{Ind}
\DeclareMathOperator{\Tr}{Tr}
\DeclareMathOperator{\Lie}{Lie}
\DeclareMathOperator{\Vect}{Vect}
\DeclareMathOperator{\Span}{Span}
%%%%%%%%%%%%%%%%%%%%%%%%%%%%%%%%%%%%%%%%%%%%%%%%%%%%%%%%%%%%%%%%%%%%%
\newcommand\restr[2]{{% we make the whole thing an ordinary symbol
  \left.\kern-\nulldelimiterspace % automatically resize the bar with \right
  #1 % the function
  \vphantom{\big|} % pretend it's a little taller at normal size
  \right|_{#2} % this is the delimiter
  }}
%%%%%%%%%%%%%%%%%%%%%%%%%%%%%%%%%%%%%%%%%%%%%%%%%%%%%%%%%%%%%%%%%%%%%

\newtheorem{theorem}{Theorem}[section]
\newtheorem*{theorem*}{Theorem}
\newtheorem{proposition}[theorem]{Proposition}
\newtheorem*{proposition*}{Proposition}
\newtheorem{conjecture}[theorem]{Conjecture}
\newtheorem*{conjecture*}{Conjecture}
\newtheorem{lemma}[theorem]{Lemma}
\newtheorem*{lemma*}{Lemma}
\newtheorem{corollary}[theorem]{Corollary}
\newtheorem*{corollary*}{Corollary}
\newtheorem{definition}[theorem]{Definition}
\newtheorem*{definition*}{Definition}
\newtheorem{example}[theorem]{Example}
\newtheorem*{example*}{Example}
\newtheorem{exercise}[theorem]{Exercise}
\newtheorem*{exercise*}{Exercise}
\newtheorem{remark}[theorem]{Remark}
\newtheorem*{remark*}{Remark}
\newtheorem{question}[theorem]{Question}
\newtheorem*{question*}{Question}
\newtheorem{claim}[theorem]{Claim}
\newtheorem*{claim*}{Claim}

\usepackage{tikz}
\tikzset{>=latex}
\usetikzlibrary{calc}
\usetikzlibrary{backgrounds}
\usetikzlibrary{patterns,decorations.pathreplacing}
\usetikzlibrary{spy}
\usepackage{pgfplots}
\pgfplotsset{compat=1.12}
\usepgfplotslibrary{colormaps}
\usepgfplotslibrary{patchplots}
\usepgfplotslibrary{fillbetween}
\usepackage{array} % for \newcolumntype macro
\newcolumntype{C}{>{{}}c<{{}}} % for columns with binary operators
\newcommand\vv{\multicolumn{1}{c}{\vdots}}

\newcommand{\rubric}[1]{{\color{blue}{#1}}}

\begin{document}
\title{MATH223 Homework 4\\ (due Friday, Oct/18, 11:59pm)}
\author{}
\date{}
\maketitle
\pagenumbering{gobble}

\begin{enumerate}
\item (3 marks) If $S\subset V$ is an arbitrary subset (not necessarily a subspace), and $T:V\to V$ is a transformation (any function, not necessarily linear), we define the image of $S$ under $T$ to be
  \[
    T(S)=\left\{ T(\vec v) \; | \; \vec v \in S\right\}.
  \]
  Consider the following figure:
        \newcommand{\transformedpic}[2]{
    \begin{tikzpicture}[scale=.8]
      \draw[gray!50!white, very thin] (-2.1,-2.1) grid (2.1,2.1);
      \draw[->] (-2.1,0) -- (2.3,0);
      \draw[->] (0,-2.1) -- (0,2.3);
      \node at (-1.6,1.5) {#2};
      \node at (1.7,-0.3) {$x$} ;
      \node at (-0.3,1.7) {$y$} ;
      \draw[cm={#1},blue, very thick] (0,0) -- (0,1) -- (1,1) -- (1,0) -- (0,0)  (0,1) -- (1,0);
    \end{tikzpicture}
  }
  \begin{center}
    \transformedpic{1,0,0,1,(0,0)}{$\mathbf{S}$}
    \transformedpic{0,1,-1,0,(1,0)}{$A$}
    \transformedpic{1,1,0,1,(0,0)}{$B$}
    \transformedpic{2,0,0,2,(-1,-1)}{$C$}
    \transformedpic{2,0,0,1,(0,0)}{$D$}
    \transformedpic{1,1,1,1,(0,0)}{$E$}
    \transformedpic{1,1,-1,1,(0,0)}{$F$}
    \begin{tikzpicture}[scale=.8]
      \draw[gray!50!white, very thin] (-2.1,-2.1) grid (2.1,2.1);
      \draw[->] (-2.1,0) -- (2.3,0);
      \draw[->] (0,-2.1) -- (0,2.3);
      \node at (-1.6,1.5) {$G$};
      \node at (0,0) [circle, fill , blue, inner sep=1pt]{};
    \end{tikzpicture}
  \end{center}
  On it you see an original subset of $\bR^2$ (in blue, named \textbf{S}) and the images of \textbf{S} under some transformation (a different one for each of A, B, \ldots ,G). For each of the images:
  \begin{itemize}
  \item Determine if the corresponding transformation can be linear or not.
  \item For the transformations that can be linear, find a matrix that could be the matrix of the transformation with respect to the standard basis of $\bR^2$.
  \item For the transformations that can not be linear, explain why they can not be.
  \end{itemize}
\item Let $\frac{d}{dx}$ denote differentiation, $\operatorname{mult}_{x}$ denote multiplication by $x$. We have seen that we can consider these as linear operators on $\mathcal{P}(\mathbb{R})$.\footnote{These two operators are the generators for the \href{https://en.wikipedia.org/wiki/Weyl_algebra}{Weyl algebra}, introduced by Herman Weyl to study the Heisenberg uncertainty principle in quantum mechanics.} To clarify: a linear operator is a linear map from a vector space to itself.
  \begin{enumerate}
  \item (1 mark) If we restrict the domain of the linear map $\operatorname{mult}_{x}$ to the subspace $\mathcal{P}_m(\mathbb{R})$, what is
    \[
      \operatorname{range}\left(\operatorname{mult}_{x}\right)?
    \]
  \item (1 mark) Consider both $\operatorname{mult}_{x}$ and $\frac{d}{dx}$ as linear maps from $\mathcal{P}_m(\mathbb{R})$ to $\mathcal{P}(\mathbb{R})$. Prove that    
    \[
      \operatorname{range}\left(\frac{d}{dx}\operatorname{mult}_{x}\right)\subseteq \mathcal{P}_m(\mathbb{R}),
    \]
    so $\frac{d}{dx}\operatorname{mult}_{x}$ can be considered as an operator on $\mathcal{P}_m(\mathbb{R})$.
  \item(2 marks) \label{part:commutator} Find a simpler expression for the linear operator
    \[
      \frac{d}{dx}\operatorname{mult}_{x}-\operatorname{mult}_{x}\frac{d}{dx}.
    \]
    (\textbf{Hint:} If you consider this as an operator on $\mathcal{P}_m(\mathbb{R})$ then you may be able to use a Lemma)
  \item (1 mark) Use your result above to simplify the expression
    \[
      \left(\frac{d}{dx}\right)^3\operatorname{mult}_{x} - \operatorname{mult}_{x} \left(\frac{d}{dx}\right)^3-2\left(\frac{d}{dx}\right)^2.
    \]    
   \item (1 mark) \label{part:matrix}Find the matrix of $\frac{d}{dx}\operatorname{mult}_{x}$ (as an operator on $\mathcal{P}_m(\mathbb{R})$) with respect to the basis $1,x,x^2,\ldots , x^m$.
   \item (1 mark) Use parts (\ref{part:commutator}) and (\ref{part:matrix}) to find the matrix of $\operatorname{mult}_{x}\frac{d}{dx}$ (with respect to the basis $1,x,x^2,\ldots , x^m$) without doing any more direct computation or matrix entries.
  \end{enumerate}
\item (2 marks, this is Exercise 11 in Axler 3A) Suppose $V$ is finite-dimensional and $T\in \mathcal{L}(V)$. Prove that $T$ is a scalar multiple of the identity if and only if $ST=TS$ for every $S\in \mathcal{L}(V)$.\footnote{This is a special case of \href{https://en.wikipedia.org/wiki/Schur\%27s_lemma}{Schur's Lemma}, an important result in representation theory.}
\item Recall the notion of direct sums of vector spaces (definition 1.41 in Axler 1C). Let $U,V$ be subspaces of some vector space, and let $W$ be finite-dimensional vector space. 
  \begin{enumerate}
  \item (1 mark) Prove that if $\vec u_1\ldots \vec u_m$ is a basis for $U$ and $\vec v_1,\ldots \vec v_n$ is a basis for $V$, then
    \[
      \vec u_1,\ldots ,\vec u_m, \vec v_1,\ldots ,\vec v_n
    \]
    is a basis for $U\oplus V$.
  \item (1 mark) Given linear maps $S:U\to W, T:V\to W$, we define the direct sum of the maps $(S\oplus T):(U\oplus V)\to W$ as follows:
  \[
    (S\oplus T)(\vec u + \vec v) = S(\vec u) + T(\vec v).
  \]
  Explain why the above rule uniquely determines the linear transformation (\textbf{Hint:} your explanation should involve clarifying the roles of $\vec u$ and $\vec v$).
\item (2 marks) Let $\vec w_1,\ldots, \vec w_d$ be a basis for $W$. Find the matrix of $(S\oplus T)$ with respect to the bases $\vec u_1,\ldots ,\vec u_m,\vec v_1,\ldots ,\vec v_n$ and $\vec w_1,\ldots, \vec w_d$. in terms of the matrices of $S$ and $T$ (with respect to the bases $\vec u_1,\ldots ,\vec u_m$ and $\vec v_1,\ldots ,\vec v_n$).
\item (1 mark) Given linear maps $Q:W\to U$ and $R:W\to V$, we define the direct sum of the maps $
  \begin{pmatrix}
    Q \\
    \oplus \\
    R
  \end{pmatrix}: W\to (U\oplus V)$ as follows:
  \[
  \begin{pmatrix}
    Q \\
    \oplus \\
    R
  \end{pmatrix}(\vec w)=Q(\vec w)+R(\vec w).
  \]
  Find the matrix of $
  \begin{pmatrix}
    Q \\
    \oplus \\
    R
  \end{pmatrix}$ with respect to the bases $\vec w_1,\ldots ,\vec w_d$ and $\vec u_1,\ldots ,\vec u_m,\vec v_1,\ldots ,\vec v_n$.2
\item (1 mark) Find the null space of $
  \begin{pmatrix}
    Q \\
    \oplus \\
    R
  \end{pmatrix}$ in terms of $\operatorname{null}(Q)$ and $\operatorname{null}(R)$.
  \end{enumerate}
\item (2 marks, this is Exercise 5 in Axler 3C) Suppose $V$ and $W$ are finite-dimensional and $T\in \mathcal{L}(V,W)$. Prove that there exist a basis of $V$ and a basis of $W$ such that with respect to these bases, all entries of $\mathcal{M}(T)$ are $0$ except that the entries in row $k$, column $k$ equal $1$ if $1\leq k \leq \dim \operatorname{range} T$.
\end{enumerate}


\bibliography{c:/Dropbox/TeX/mybibliography}{}
% \bibliography{/media/bazse/storage/Dropbox/TeX/mybibliography}{}
\bibliographystyle{alpha}

\end{document}
